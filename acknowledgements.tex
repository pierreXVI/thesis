\chapter{Remerciements}

  \paragraph{}
  Même si trois années passent vite, une thèse est une véritable période de vie, durant laquelle beaucoup de personnes se retrouvent investies.
  Je tiens à remercier ici ces personnes qui m'auront accompagné, aidé et soutenu durant cette période.

  \paragraph{}
  Je commence par remercier les membres du jury qui ont accepté leur rôle.
  Je remercie en particulier les rapporteurs.
  Comme j'ai eu l'occasion de leur dire, leurs rapports sont d'un grand intérêt car ils permettent de prendre du recul sur son propre travail, en donnant un point de vue critique mais surtout extérieur sur la thèse.
  Enfin, je remercie Xavier Vasseur qui n'était pas à priori lié à cette thèse mais qui y a contribué, en répondant à mes questions, me fournissant des références et enfin en acceptant le rôle de juré.

  \paragraph{}
  Ma thèse s'étant déroulée initialement à Toulouse puis à Châtillon, je n'ai jamais été physiquement avec mes deux encadrants en même temps, à part de rares occasions.
  Cependant, je ne me suis jamais senti seul dans mon travail.
  Que ce soit au téléphone, par chat, par mail ou lors du point hebdomadaire, vous étiez toujours disponible.
  Pour ceci, Guillaume et Lionel, je vous suis extrêmement reconnaissant et je ne peux que recommander un tel encadrement de thèse.
  Je pense aussi que vos qualités d'encadrants sont liées à vos personnalités : en dehors du contexte professionnel, nous nous sommes toujours bien entendus, ce qui facilite d'autant plus nos échanges.
  Merci à tous les deux, à Guillaume pour ton point de vue optimiste lorsque je n'en avais pas, à Lionel de m'avoir laissé te harceler par chat et au téléphone.
  Je suis convaincu que vous avez contribué au bon déroulement de cette aventure, et pour cela merci encore.

  \paragraph{}
  J'ai donc commencé ma thèse dans l'unité HEAT avant de déménager dans l'unité PLM.
  Ce fut pour moi l'occasion de rencontrer davantage de personnes qui ont ajouté leur pierre à l'édifice.
  Certes, cela veut en fait dire que j'avais d'avantage de personnes à aller embêter avec mes questions, mais j'avais toujours de l'aide en retour.
  Je remercie donc les membres de ces deux unités, et même plus largement les membres du DMPE, qui m'ont aidé durant ces trois ans.

  \paragraph{}
  Parmi eux, je remercie en particulier Jean-Michel Lamet.
  C'est parce qu'il s'est intéressé à ma thèse que nous avons pu trouver un avantage immédiat à l'utilisation de mes travaux.
  Merci pour le temps que tu m'as consacré, à re-travailler ton modèle, échanger avec moi, et même relire une partie de ce mémoire !

  \paragraph{}
  Plus largement, je tiens à remercier l'ONERA et son personnel.
  Je m'adresse ici aux personnes qui ont n'ont pas contribué à ma thèse d'un point de vue scientifique, mais qui ont tout de même apporté leur aide.

  \paragraph{}
  En-dehors du personnel ONERA, je remercie les différentes personnes qui ont fait partie du support CEDRE.
  Je pense notamment à Julien R. qui m'impressionna par ses compétences et son investissement.
  Je pense également à Alexandre et Christ, avec qui j'ai pu râler sur les problèmes et les bugs que je rencontrais.
  À ces personnes s'ajoutent Julien V., Yacine, ou encore mes collègues doctorants.
  Merci à vous tous pour ces moments d'échanges parfois scientifique, parfois moins.

  \paragraph{}
  Il ne faut pas oublier le plus grand contributeur de cette thèse, sans qui ce mémoire n'existerait pas aujourd'hui.
  Merci Zhao pour ton ordi !

  \paragraph{}
  Je remercie enfin ma famille pour son soutien, et en particulier mes parents.
  Vous dites être fiers de moi, mais je suis comme je suis de par votre éducation et votre amour, merci.
  Merci Manon, de partager ma vie, et d'en faire un bonheur.

\vspace*{\stretch{1}}
\begin{flushright}
    \textit{Pour mamie Raymonde, qui disait que plus tard, je serai ingénieur.}
\end{flushright}
