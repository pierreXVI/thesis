\section{Résumé}

\paragraph{}
Cette thèse s'intéresse aux performances de l'intégration temporelle du solveur CEDRE sur des problèmes stationaires.
Il s'agit d'une plateforme logicielle visant la résolution des problèmes multifluides pour des applications en énergétique à échele industrielle.
CEDRE est en effet composé de différents solveurs, chacun dédié à un ensemble de phénomènes physique.
Plus précisément, nous regardons comment améliorer la chaine d'intération temporelle du point de vue de la rapidité, de la robustesse et de la convergence.
Pour des raisons de stabilité nous nous intéressons à des méthodes implicites, et en particulier la méthode d'Euler implicite.
Les méthodes implicites nécessitent la résolution de problèmes non-linéaires, qui nécessitent à leur tour la résolution de problèmes linéaires.
Le passage de l'un à l'autre se fait par la présence de la Jacobienne des fonctions intervenant dans les problèmes non-linéaires.
Une méthode de Krylov est déjà existante dans CEDRE pour l'inversion de systèmes linéaires: la méthode GMRES.
Nous utilisons le fait qu'une telle méthode n'a pas explicitement besoin de la matrice pour l'inverser et mettons en place une méthode Jacobian-Free Newton--Krylov.
Le but est d'améliorer la précision de la matrice Jacobienne utilisée, en espérant que cela améliorera la précision globale de l'intégration temporelle.
Ceci est justifié par le fait qu'avant cette thèse la Jacobienne utilisée est très approximée et ne prends pas en compte becoup des modélisations fines du solveur, comme les reconstruction MUSCL, les termes sources turbulents ou chimiques, ainsi que les relations de fermetures thermodynamiques qui peuvent être complexes dans CEDRE.
Une implémentation d'une méthodologie sans-matrice est mise en place, de manière générique de sorte que tout solveur de CEDRE puisse utiliser cette formulation.
Cela ouvre de plus la porte à une résolution implicite couplée de solveurs, chose qui n'était pas permise avec la structure actuelle de CEDRE.

Pour attester de l'intéret de la méthode JFNK, elle est comparée aux méthodes préexistantes de CEDRE sur des applications typiques de complexité croissantes.
Ces applications sont choisies affin de représenter les fonctionalités du solveur.
Nous nous intéressons tout d'abord à un profil d'aile sans incidence dans un écoulement transsonique turbulent pour lequel la couche limite est modélisée au niveau de la paroi.
Nous regardons ensuite le même profil mais en incidence avec un mailage beaucoup plus fin pour capturer la couche limite turbulente autour du profil.
Dans les deux cas, nous comparons la méthode JFNK à la méthode plus classique utilisée dans CEDRE.
Nous regardons la norme des résidus pour montrer que la méthode utilisant une matrice plus précise converge mieux car elle prend en compre plus fidèlement l'ensemble des modèles physique.

Nous nous intéressons ensuite à une sphère solide dans un écoulement hypersonique.
Ce cas correspond aux problématiques de rentrée atmosphérique auquelles fait face CEDRE.
En raion de la haute énergie du cas, de fortes réactions chimiques ont lieux en aval du choc, face à la sphère.
La même approche est mise en place pour montrer que la méthode JFNK permet une meilleure convergence du calcul.
Puis, nous utilisons un autre modèle physique récemment mis en place afin de représenter les déséquilibres thermodynamiques ayant lieu dans de telles conditions en aval du choc.
Ce modèle étant nouveau, la Jacobienne associée n'est pas disponible, et en conséquence les méthodes implicites ne le sont pas non plus.
La méthode JFNK n'utilisant pas la matrice l'est cependant, et est donc comparée à une méthode explicite, seule solution disponible aux utlisateurs pour réaliser des calculs avec ce modèle.
Nous montrons donc qu'en plus d'améliorer la convergence, la méthode JFNK permet d'améliorer le temps de calcul sur ces applications.

Dans un second temps, nous élargissons le contexte en nous intéressant aux méthodes d'intégration exponentielles, cette fois avec le solveur JAGUAR.
Ce changement de solveur est justifié par la plus grande précision apportée par la méthode des Différences Spectrales qu'il utilise comme schéma de discrétisation spatiale, nécessaire à l'analyse de ces nouveaux schémas également très précis.
Nous choisissons, implémentons et analysons un ensemble de méthodes exponentielles, en comparaison à des méthodes déja présentes, sur plusieurs cas pour montrer leur intérêt.
