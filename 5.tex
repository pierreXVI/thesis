\chapter{Exponential integration methods in JAGUAR}

  \paragraph{}
  We want to analyse time exponential integration methods in a framework that gives us access to high-order spatial discretisation methods.
  It is possible to use high-order Finite Volume methods, but this means to use larger stensils which hurts parallelism.
  In CEDRE, users often stop at second-order methods, so we need to use an other solver.
  As the spatial discretisation method does not plays a direct role in the time integration and the work done in this thesis, we can step out from the Finite Volume framework.
  Furthermore, using a less complex solver will also help us to try and develop new methods more easily than what we already did with CEDRE.
  This is why we decided to accomplish our analysis of exponential time integration methods with the solver JAGUAR.


  \section{JAGUAR: a spectral difference solver}

    \paragraph{}
    JAGUAR means proJect of an Aerodynamic solver using General Unstructured grids And high ordeR schemes.
    It is a reactive Navier--Stokes solver originally developed at the European center for research and advanced training in scientific computing: CERFACS.
    It is made for unstructured grids, and uses a Large Eddy Simulation model to solve turbulence.
    Its particularity is that is use a spectral spatial discretisation method: the Spectral Difference method.


    \subsection{The Spectral Difference method}

      \paragraph{}
      With the Spectral Difference method, the solution is represented by a polynomial of degree $p$ inside each cell.
      It means that in the partial differential equation
      \begin{equation}
        \frac{\partial u}{\partial t} + \nabla \cdot F\left(u\right) = 0 \ ,
      \end{equation}
      $u$ is a $p$-degree polynomial of the position variables, where the polynomial coefficients are functions of time.
      Then $F\left(u\right)$ has to be a $p + 1$-order polynomial of the position variables.
      The key to the Spectral Difference method is how to compute a $p + 1$-order $F\left(u\right)$ from a $p$-order $u$.
      This method uses key elements that were first mentioned by \cite{Kopriva1996}, and was later developed by \cite{LiuVinokurWang2006}.

      \paragraph{}
      To better understand how this method works, we take a 1D cell: the segment $\left[0, 1\right]$.
      Because the following is done at a fixed time we will drop the dependency on the time, but the solution and the coefficients are in reality functions of the time, not scalars.
      The solution $u$ inside this segment is then $u\left(x\right) = \sum_{i=0}^p a_ix^i$.
      Using Lagrange interpolation polynomials, it is equivalent to use the $p + 1$ coefficients $a_i$ or the $p + 1$ values $u\left(x_i\right)$ computed at the distincts points $x_i \in \left[0, 1\right]$ called \emph{solution points}.
      The solution is characterised by either one of those two sets.




    \subsection{Exponential integration methods in JAGUAR}






  \section{Analysis of exponential time integration methods}

    \subsection{Order: convected inviscid isentropic vortex}
    \subsection{Robustness: Taylor--Green vortex}
    \subsection{Industrial application: LS89}
