\documentclass[a4paper]{report}

\usepackage[utf8]{inputenc}
\usepackage[english]{babel}
\usepackage[T1]{fontenc}

\usepackage{geometry}
\usepackage{amssymb}
\usepackage{amsmath}
\usepackage{ifthen}
\usepackage{caption}
\usepackage{graphicx}
\usepackage{subcaption}
\usepackage{minted}
\usepackage{siunitx}
\usepackage[version=4]{mhchem}
\usepackage[dvipsnames]{xcolor}
\usepackage[ED=MEGEP-DyF, Ets=ISAE]{tlsflyleaf/tlsflyleaf}
\usepackage{array}
\usepackage{tcolorbox}


\renewcommand*{\vec}[1]{\underline{#1}}
\newcommand*{\mat}[1]{\vec{\vec{#1}}}
\newcommand{\norm}[2][]{\left\|{#2}\right\|\ifthenelse{\equal{#1}{}}{}{_{{#1}}}}
\newcommand{\RKBar}{\rule[-1.1ex]{0pt}{0pt} \\ \hline \rule{0pt}{2.6ex} &}
\newcommand{\transpose}[1]{{#1}^T}
\newcommand{\krylov}[2][]{\ifthenelse{\equal{#1}{}}{\mathcal{K}_{#2}}{\mathcal{K}_{#2}\left({#1}\right)}}

\newcolumntype{M}[1]{>{\centering\arraybackslash}m{#1}}

\usepackage[pdfpagelabels]{hyperref}
% \hypersetup{
%     colorlinks,
%     linkcolor=black,
%     citecolor=black,
%     % filecolor=black,
%     % urlcolor=black
% }

\newcommand{\GP}[1]{\textbf{\color{BurntOrange}{#1}}}
\newcommand{\LM}[1]{\textbf{\color{ForestGreen}{#1}}}
\newcommand{\PS}[1]{\textbf{\color{Red}{#1}}}
\renewcommand{\GP}[1]{}
\renewcommand{\LM}[1]{}
\renewcommand{\PS}[1]{}


\title{\textbf{\large Méthodologies permettant l'obtention efficace de solutions multi-physiques stationnaires pour des applications en énergétique}}
\author{Pierre Seize}
\defencedate{TBD}
\lab{Office National d'Études et de Recherches Aérospatiales - Département Multi-Physique pour l'Énergétique}
% \\ OU \\ ISAE-ONERA EDyF Energétique et Dynamique des Fluides}
\njudge{7}
\nboss{2}
\nreferee{2}
\makesomeone{judge}{1}{Jocelyne ERHEL}{Directrice de recherche}{Membre du jury}
\makesomeone{judge}{2}{Pierre-Henri MAIRE}{Directeur de recherche}{Membre du jury}
\makesomeone{judge}{3}{Xavier VASSEUR}{Docteur, HDR}{Membre du jury}
\makesomeone{judge}{4}{Rodolphe TURPAULT}{Professeur des Universités}{Rapporteur}
\makesomeone{judge}{5}{Vincent PERRIER}{Chargé de recherche, HDR}{Rapporteur}
\makesomeone{judge}{6}{Guillaume PUIGT}{Directeur de recherche}{Directeur de thèse}
\makesomeone{judge}{7}{Lionel MATUSZEWSKI}{Docteur}{Co-Directeur de thèse}
\makesomeone{referee}{1}{Rodolphe TURPAULT}{Professeur des Universités}{}
\makesomeone{referee}{2}{Vincent PERRIER}{Chargé de recherche}{}
\makesomeone{boss}{1}{Guillaume PUIGT}{}{}
\makesomeone{boss}{2}{Lionel MATUSZEWSKI}{}{}


\begin{document}
\maketitle
\tableofcontents
\addcontentsline{toc}{chapter}{Table of contents}
\pagebreak

\chapter{Analyse des méthodes existantes}
  \section{Brève introduction aux méthodes d'intégration spatiale de CEDRE}
    \subsection{Finite Volumes method}
    \subsection{Gradient reconstruction methods}
      \subsubsection{The k-exact method}
      \subsubsection{The Multislope method}
  \section{Time integration methods}
    \subsection{Analyse des méthodes}
      \subsubsection{Consistance et ordre}
      \subsubsection{Stabilité}
    \subsection{Méthodes implicites}
      \subsubsection{Méthode d'Euler implicite}
      \subsubsection{Méthodologie des méthodes implicites}




\chapter{Implémentation d'une méthode JFNK dans CEDRE}

\chapter{Analyse de la méthode JFNK dans CEDRE}
  \section{Comparaison entre la matrice jacobienne explicite et la formulation sans matrice}
    \subsection{Sphère hypersonique}
    \subsection{Profil RAE dans un écoulement transsonique turbulent}
  \section{Utilisation de la formulation sans matrice sur un nouveau modèle de fluide}
    \subsection{Modèle Multi-températures}
    \subsection{Sphère hypersonique}
      \subsubsection{Modèle réactif complet}
      \subsubsection{Modèle réactif simplifié}


\chapter{Etude des méthode exponentielles}
  \section{Présentation de \emph{Jaguar}}
    \subsection{Méthode des Différences Spectrales}
    \subsection{Méthodes Exponentielles}
  \section{Analyse des méthodes exponentielles}
    \subsection{Analyse de l'ordre : convection d'un tourbillon isentropique}
    \subsection{Analyse de la précision : tourbillon de Taylor--Green}








\pagebreak
\bibliography{bibliography.bib}
\bibliographystyle{ieeetr}
\addcontentsline{toc}{chapter}{Bibliographie}

\end{document}
