\documentclass[a4paper]{report}

\usepackage[utf8]{inputenc}
\usepackage[english]{babel}
\usepackage[T1]{fontenc}

\usepackage{geometry}
\usepackage{amssymb}
\usepackage{amsmath}
\usepackage{ifthen}
\usepackage{caption}
\usepackage{graphicx}
\usepackage{subcaption}
\usepackage{minted}
\usepackage{siunitx}
\usepackage[version=4]{mhchem}
\usepackage[dvipsnames]{xcolor}
\usepackage[ED=MEGEP-DyF, Ets=ISAE]{tlsflyleaf/tlsflyleaf}
\usepackage{array}
\usepackage{tcolorbox}


\renewcommand*{\vec}[1]{\underline{#1}}
\newcommand*{\mat}[1]{\vec{\vec{#1}}}
\newcommand{\norm}[2][]{\left\|{#2}\right\|\ifthenelse{\equal{#1}{}}{}{_{{#1}}}}
\newcommand{\RKBar}{\rule[-1.1ex]{0pt}{0pt} \\ \hline \rule{0pt}{2.6ex} &}
\newcommand{\transpose}[1]{{#1}^T}
\newcommand{\krylov}[2][]{\ifthenelse{\equal{#1}{}}{\mathcal{K}_{#2}}{\mathcal{K}_{#2}\left({#1}\right)}}

\newcolumntype{M}[1]{>{\centering\arraybackslash}m{#1}}

\usepackage[pdfpagelabels]{hyperref}
% \hypersetup{
%     colorlinks,
%     linkcolor=black,
%     citecolor=black,
%     % filecolor=black,
%     % urlcolor=black
% }


\title{\textbf{\large Méthodologies permettant l'obtention efficace de solutions multi-physiques stationnaires pour des applications en énergétique}}
\author{Pierre Seize}
\defencedate{13 mars 2023}
\lab{Office National d'Études et de Recherches Aérospatiales - Département Multi-Physique pour l'Énergétique}
\njudge{7}
\nboss{2}
\nreferee{2}
\makesomeone{judge}{1}{Pierre-Henri MAIRE}{Directeur de recherche}{Président du jury}
\makesomeone{judge}{2}{Jocelyne ERHEL}{Directrice de recherche}{Membre du jury}
\makesomeone{judge}{3}{Xavier VASSEUR}{Docteur, HDR}{Membre du jury}
\makesomeone{judge}{4}{Rodolphe TURPAULT}{Professeur des Universités}{Rapporteur}
\makesomeone{judge}{5}{Vincent PERRIER}{Chargé de recherche, HDR}{Rapporteur}
\makesomeone{judge}{6}{Guillaume PUIGT}{Directeur de recherche}{Directeur de thèse}
\makesomeone{judge}{7}{Lionel MATUSZEWSKI}{Docteur}{Co-Directeur de thèse}
\makesomeone{referee}{1}{Rodolphe TURPAULT}{Professeur des Universités}{}
\makesomeone{referee}{2}{Vincent PERRIER}{Chargé de recherche}{}
\makesomeone{boss}{1}{Guillaume PUIGT}{}{}
\makesomeone{boss}{2}{Lionel MATUSZEWSKI}{}{}


\begin{document}
% \maketitle
% \tableofcontents
% \addcontentsline{toc}{chapter}{Table of contents}


% \chapter*{Introduction}
\addcontentsline{toc}{chapter}{Introduction}


  \section*{General context}

    \paragraph{}
    This work is rooted in the field of numerical simulation of fluid dynamics, applied to \PS{energy/energetics}.
    This field gathers many industrial actors (the French Directorate General of Armaments, ArianeGroup, Safran, Airbus, etc) as well as academics (ONERA, CERFACS, DLR, VKI, universities, etc).
    It is interested in how to simulate a fluid flow with a computer, trying to represent the physical reality faithfully.
    The various actors in this field need to be able to access certain physical quantities associated with specific phenomena and operating regimes.
    These regimes are often not feasible on our scale, due to material or financial limitations.
    Examples include the study of icing on the wing of an aircraft, which is experimentally feasible but represents an imposing budget for the aircraft manufacturer, or the study of heat transfer in an atmospheric reentry capsule, which is much more difficult to achieve experimentally.
    To overcome these limitations, numerical simulation is the best option, as it allows such a case study to be modelled by the execution of a computer program, and to obtain a large set of data that will be analysed afterwards to answer the desired questions.

    \paragraph{}
    The analysis of physics usually produces a set of equations, often partial differential equations, representing the real system that one wishes to study.
    Algorithms are then required to determine the fluid flow from these equations, in the working domain, as a function of time.
    Thus, to obtain the desired quantities, the physical system is integrated in time using mathematical algorithms to obtain its evolution in time.
    Often, the expected result is not the complete temporal evolution of the system, but only its equilibrium state.
    This is called a steady numerical simulation or steady computation.
    Steady computations are opposed to unsteady computations that aim to accurately describe the system temporal evolution.

    \paragraph{}
    For all computational fluid dynamics users, the most crucial issue is to achieve a satisfying compromise between computational cost and accuracy of the results.
    Indeed, a quick and inexpensive computation tends to be not very faithful to physics, while an accurate computation tends to use more computational ressources and time.
    For steady computations, speed corresponds to getting the final steady state at a low time cost, for both computational time and the time it actually took to a user.
    It results in a compromise between methods that are expensive in terms of computational ressources and take a long time but give accurate results that are close to the physical reality, and faster methods that saves ressources but give lesser quality results.
    A software developer working on a numerical simulation tool needs to choose methods and algorithms to obtain a compromise that he considers satisfactory.
    The final interest for a player in computational fluid dynamics is therefore to have a result that is sufficiently precise and inexpensive enough to obtain.
    An accurate result is needed to answer the questions that required the simulation.
    The search for a result that is inexpensive to obtain is motivated by questions of savings in computational cost.
    This cost is applied to the user, while he wait for the simulation to end, and to its company through the cost in computer ressources, electricity, investments in more efficient machines, etc.

    \paragraph{}
    Depending on the problem to be solved, there are more or less suitable algorithms and methods.
    Performant methods were originally developed to answer needs from the aerodynamics community.
    In the case of energetic and multi-physics problems, the methods issued from the more traditional aerodynamics are limited by the coupling  between the different physics which have their distinct characteristic times.
    Thus, the algorithms used for the numerical simulation of classical fluid dynamics, i.e. concentrated on the resolution of the Navier--Stokes equations, are not necessarily the most adapted to a simulation in the energetic domain.
    The involvement of several distinct physical phenomena imposes constraints on the choice and use of algorithms.
    Consequently, it would be advisable to adapt or replace the algorithms involved in the time integration for multiphysics problems.

  \section*{Études}

    \paragraph{}
    Si la simulation numérique s'est grandement développée dans le domaine l'aéronautique, elle ne s'est pas tant adaptée au domaine de la multi-physique, et de nombreux codes industriels se contentent de réutiliser les mêmes algorithmes.
    C'est l'exemple du code CEDRE, développé à l'ONERA par le Département Multi-Physique pour l'Énergétique.
    Ce code constitue une plateforme regroupant plusieurs solveurs pour intégrer plusieurs physiques : chaque solveur est dédié à son modèle physique.
    On compte alors un solveur pour la résolution des écoulements compressibles, multi-fluides, réactifs et turbulents, deux solveurs pour le calcul de phase dispersée (gouttes, cristaux, particules) en approche eulérienne et lagrangienne respectivement, un solveur dédié au calcul des films liquides, un solveur dédié au rayonnement, ...
    Des travaux on été fais pour mettre en place une intégration temporelle adaptée aux problèmes résolus par CEDRE.
    C'est par exemple le cas de \cite{Selva1998}.
    Un travail sur l'intégration temporelle a mené au développement de méthodes d'intégrations implicites pour l'intégration des problèmes stationnaires, et au développement d'une méthode GMRES pour la résolution des systèmes linéaires.
    Ainsi, CEDRE constitue en fait un solveur global adapté aux problèmes multi-physiques.
    L'utilisateur peut choisir parmi un panel de méthodes d'intégrations pour obtenir une méthode adaptée à son problème.
    Cependant, la couplage faible entre solveurs entache la convergence vers l'état stationnaire et le choix dans les méthodes d'intégration est limité, en comparaison à ce qu'on peut trouver dans la littérature.
    C'est du moins l'avis des acteurs du code CEDRE, c'est à dire ses développeurs et ses utilisateurs, qui aimeraient des méthodes plus robustes pour pouvoir utiliser CEDRE sur des problèmes plus raides, et convergeant plus vite pour économiser en coût de calcul.

    \paragraph{}\PS{TODO}
    Du coté de la recherche, cependant, de nombreux efforts ont été réalisés dans le sens de la multi-physique, mais ne sont pas encore sorti du cadre académique.
    C'est par exemple le cas de \cite{WongKwokHorneEtAl2019}, qui se sont intéressés à l'intégration temporelle d'équations couplées.
    Ils ont conçu une méthode d'intégration adaptée aux équations couplées, qui est une évolution d'une méthode du point fixe avec en plus une étape d'une méthode de Newton.
    Ils ont ensuite comparés cette méthode à la méthode du point fixe standard, plus généralement utilisée pour une résolution couplée.
    En mettant en place leur méthode sur deux problèmes de couplage simple, ils ont enfin montré l’intérêt de leur méthode par rapport à la méthode de base.
    Si cette méthode se prête bien à leur calculs, elle n'est cependant mise en place que pour des problèmes simples, moins complexes que les problèmes multi-physiques que CEDRE souhaite résoudre.
    De plus, les tests réalisés sont sur des problèmes à l'échelle académique, et non industrielle.

    \paragraph{}
    Parallèlement, des outils utilisés dans le cadre de la simulation aérodynamique pourraient s'avérer intéressants pour des problèmes multi-physiques.
    La méthode JFNK est déjà bien utilisée dans la simulation numérique des équations de Navier-Stokes.
    Dans \cite{ParkNourgalievMartineauEtAl2009}, un mécanisme d'intégration temporelle est mis en place autour d'une formulation JFNK.
    Un préconditionnement basé sur la physique est développé pour résoudre plus précisément le problème linéaire.
    Cette intégration temporelle est ensuite testée sur un problème de cavitée carée thermiquement entrainée.
    Cependant, cette méthode ne s'intéresse seulement aux équations de Navier-Stokes et n'est pas directement appliquable aux problèmes plus généraux de l'énergétique.
    De plus, elle n'est testée que sur un problème académique 2D.

    \PS{TODO: quand parler de la distinction CEDRE / CHARME, placer \cite{ReflochCourbetMurroneEtAl2011}}

  \section*{Bilan général}

    \paragraph{}
    On voit donc que des méthodes numériques sont déjà disponibles pour résoudre des problèmes de simulation numérique.
    En particulier, il existe déjà des solveurs capables de résoudre les problèmes stationnaire multi-physiques d'échelle industrielle.
    D'un autre coté, d'autres méthodes développées dans un cadre académique ont montré leur intérêt sur des problèmes multi-physiques simples.

    Enfin, des méthodes améliorant la résolution des problèmes d'aérodynamique pourraient s'avérer intéressantes pour des problèmes multi-physiques.

    Enfin, certaines méthodes récentes ont permis ...

    Les solveurs déjà existants utilisent des méthodes anciennes, ou peu adaptées aux problèmes multi-physiques.
    Les méthodes plus performantes ne sont utilisées que sur des problèmes simplifiés, ou dans des calculs d'échelle académique.

    Il semblerait maintenant intéressant d'adapter ces nouvelles méthodes pour la résolution des problèmes multi-physiques dans un code industriel.

    \vspace{1cm}\hrule\vspace{1cm}

  \section*{C'est ce qui justifie cette étude ...}

    \paragraph{}
    ..., elle consiste à améliorer la convergence, la rapidité et la robustesse de l'intégration temporelle de la plateforme CEDRE sur les problèmes stationnaires multi-physiques en ajoutant des méthodes numériques non utilisées dans la simulation numérique industrielle.


  \section*{Démarche}

    \paragraph{}
    L'objectif du chapitre 1 a été de développer une formulation JFNK pour AMÉLIORER QUOI DE l'intégration temporelle dans le code CEDRE.
    JUSTIFICATION
    Pour cela, l'idée a été d'adapter la méthode d'intégration en s'inspirant de la bibliographie en fonction de certains critères.
    L'idée suivante a été de développer une méthode de Newton afin de résoudre le problème non linéaire sachant que celui ci est produit par les méthodes d'intégration implicites utilisées lors de la résolution des problèmes stationnaires.
    L'idée suivante a été de développer une formulation sans matrice du système afin de mieux former les systèmes linéaires et permettre un couplage fort des solveurs de CEDRE.
    L'idée suivante a été d'adapter la résolution du système linéaire en utilisant des méthodes plus modernes que la méthode actuelle, en s'inspirant de la bibliographie, afin de résoudre plus précisément le système linéaire.
    On a alors obtenu une méthodologie d'intégration temporelle utilisable pour résoudre des problèmes multi-physique stationaires.
    A ce stade, cette méthode fonctionne au sens où elle permet d'obtenir un résultat à un problème de simulation numérique, mais elle n'est pas encore caractérisée.

    \paragraph{}
    L'idée du chapitre 2 a été d'évaluer la robustesse et la convergence de la formulation JFNK sur des problèmes stationnaire en multi-physiques au sein du code CEDRE.
    Pour cela, l'idée a été de sélectionner des cas tests de complexité croissante afin de représenter un ensemble de problèmes multi-physiques.
    L'idée suivante a été de mettre en place une méthode d'évaluation des performances de l'intégration temporelle afin de caractériser sa robustesse et sa vitesse de convergence.
    L'idée suivante a été de montrer que sur ces cas la formulation améliore la robustesse et la convergence afin de valider la pertinance du choix de la méthode.
    On estime qu'on a alors développé une méthode d'intégration temporelle "efficace", c'est à dire rendant l'intégration temporelle de CEDRE plus robuste et convergeant plus rapidement sur un ensemble de problèmes multi-physiques.
    On pourrait aller plus loin mais on va plutôt regarder l'intérêt de la formulation JFNK sur des problèmes autres pouvant en tirer profit : les problèmes instationnaires à grand pas de temps.

    \paragraph{}
    L'idée du chapitre 3 est d'analyser la formulation JFNK sur les problèmes instationnaires à grand pas de temps.
    Pour cela, l'idée a été d'identifier une classe de problèmes sortant du cadre initial de la thèse mais pouvant bénéficier de la formulation JFNK, afin d'y démontrer l'intérêt de la formulation.
    L'idée suivante a été de concevoir un cas d'étude afin de de mettre en valeur l'intérêt de la formulation JFNK.
    L'idée suivante a été de montrer l’intérêt la formulation sur ce cas.



% \part{Résolution efficace des problèmes stationaires en multi-physique}

  \chapter{Analysis of existing methods}

  \paragraph{}
  As we aim to enhance the performances of the solver CEDRE on steady problems, we first have to define some notions.
  In particular, we need to explain what performance means.
  In this chapter, we will introduce the general problem, and the tools used to solve it.
  We will confirm some choices already made in the solver, and identify the \PS{features} we would like to modify or replace.

  \section{Problem setup}

    \paragraph{}
    In this part, we are going to set the mathematical framework for this study.
    We will start from a partial differential equation arising from the physical model, in the form of
    \begin{equation}\label{eq:pde}
      \frac{\partial \xi}{\partial t} + \operatorname{F}\left(\xi\right) = 0
    \end{equation}
    where the function $\operatorname{F}$ uses some space derivatives of the state variable $\xi$.
    This equation then describes the temporal evolution of the state variables $\xi$.

    \paragraph{}
    A particular class of such partial differential equations are conservative equations.
    They correspond to the case where the function $\operatorname{F}$ can be written as a divergence term.
    Finally, with a source term $\operatorname{S}$, those equations look like:
    \begin{equation}\label{eq:pde_conservative}
      \frac{\partial \xi}{\partial t} + \nabla \cdot \vec{\operatorname{f}}\left(\xi\right) = \operatorname{S}\ .
    \end{equation}
    One might notice that equation (\ref{eq:pde_conservative}) is indeed a particularisation of equation (\ref{eq:pde}), with $\operatorname{F}\left(\xi\right) = \nabla\cdot \vec{\operatorname{f}}\left(\xi\right) - \operatorname{S}$.
    Those conservative equations are the ones we will focus on in this study, as they describe the physical systems we are interested in.

    \paragraph{}
    In this work, we will talk about computational fluid dynamics.
    We are in fact mostly interested in the Navier--Stokes equation, and its variants: the reactive Navier--Stokes equation, the Reynold-averaged Navier--Stokes equation, etc.
    A simple form of this equation can be:
    \begin{equation}\label{eq:ns}
      \left\{\begin{aligned}
        &\partial_t\left(\rho         \right) &&+ \nabla\cdot\left( \rho \vec{u} \right) &&= 0 \\
        &\partial_t\left(\rho \vec{u} \right) &&+ \nabla\cdot\left( \rho \vec{u} \otimes \vec{u} + p \mat{\operatorname{Id}} \right) &&= \nabla\cdot \mat{\tau}\\
        &\partial_t\left( \rho E      \right) &&+ \nabla\cdot\left( \left(\rho E + p\right) \vec{u} \right) &&=
          \nabla\cdot\left( \mat{\tau} \cdot \vec{u} \right)
      \end{aligned}\right.
    \end{equation}
    with the \PS{relation de fermeture} $\rho E = \frac{p}{\gamma - 1} + \rho\frac{\vec{u} \cdot \vec{u}}{2}$.
    The deviatoric stress tensor $\tau$ accounts for the fluid's viscosity, and its computation depends on the model used.
    Without it, one recovers the Euler equations.
    To this simple form can be added source terms from the reactive model, source terms from the turbulence model, divergence terms from a diffusive model \PS{diffusion des particules}, etc.
    Yet it is clear that with a bit of rewriting, we can go back to the starting form (\ref{eq:pde}) and even the conservative form with source terms (\ref{eq:pde_conservative}).
    The quantity $\xi$ is no longer a scalar but a vector with the density $\rho$, each component of the momentum $\rho\vec{u}$ and the energy $\rho E$ as its components.
    Apart from this small change, the idea is the same.

    \paragraph{}
    When solving equations like (\ref{eq:pde}) numerically, one must first take a spatial domain of interest.
    Let us call this domain $\mathcal{D}$.
    As we are interested in solving equations numerically, we need to be able to represent different quantities, such as the state variable $\xi$, numerically over the domain $\mathcal{D}$ and store it in the memory of a computer.
    Therefore we need to discretise the continuous spatial domain into a finite number of cells, or elements.
    This is usually done with a mesh of the domain $\mathcal{D}$.
    First, we divide the domain $\mathcal{D}$ in a set of cells, called a mesh.
    Those cells are small disjoints volumes in 3D, faces in 2D or segments in 1D, such as their union recovers the original domain.
    Interest quantities, such as the fluid velocity, density, \dots, are then stored at each node, averaged at the center of each cell or sometimes in a more complex fashion depending on the method.
    They are no longer mathematically represented by a function of the continuous physical domain $\xi: \mathcal{D} \rightarrow \mathbb{R}$ but by a finite-sized vector $\Xi$ gathering all the information across the discretised domain.
    For some simple discretisation methods, this vector consists of the quantity evaluated at the mesh nodes or averaged at the center of the cells.
    For more complex methods, this vector consists of information used to construct the solution over the domain: polynomial coefficients, spectral decomposition coefficients, etc.
    Anyway, we no longer work in a continuous domain $\mathcal{D}$ but on a discretised one.

    \paragraph{}
    The partial differential equation (\ref{eq:pde}) transforms then into an ordinary differential equation:
    \begin{equation}\label{eq:ode}
      \frac{\partial \Xi}{\partial t} + \operatorname{G}\left(\Xi\right) = 0 \ .
    \end{equation}
    The difference here is that the function $\operatorname{G}$ is a function of a discrete vector whereas $\operatorname{F}$ was a function of a continuous function, and therefore $\operatorname{G}$ does not use any spatial derivatives.
    Thanks to the spatial discretisation method, the only derivative remaining is with regard to time.
    The rest is then up to the temporal integration method, which is the main topic of this thesis.
    We will work from equation (\ref{eq:ode}) no matter where the function $\operatorname{G}$ comes from, but sometimes understanding the origin of this function can help so we will now introduce the spatial discretisation method used in our solver.


  \section{Brief introduction to the spatial integration schemes}

    \paragraph{}
    A \emph{spatial discretisation method} is the choice of how to represent a quantity over a discretised domain, and how to compute the spatial derivative of this quantity from this representation.
    Indeed, before solving equation (\ref{eq:pde}) we need to decide how to transform the continuous model into a discretised one.
    We also have to look at how the spatial derivatives arising from equation (\ref{eq:pde}) translate in the discretised model.

    \subsection{The Finite Volume method}

      \paragraph{}
      The spatial discretisation method used in the solver CHARME is called the Finite Volume method \cite{EymardGallouetHerbin2000, Leterrier2003}.
      This method is particularly well fitted for conservatives equations such as equation (\ref{eq:pde_conservative}).
      A such equation has the property that the quantity $\xi$ is conserved: without source terms, the variation of the total quantity on $\xi$ over the domain $\mathcal{D}$ is equal to the flux $f\left(\xi\right)$ coming through the boundary $\partial\mathcal{D}$.
      In the case of the Navier--Stokes equations (\ref{eq:ns}), the density, the momentum and the energy are conserved throughout time, apart from what comes in and out of the domain.
      In a close domain where nothing comes in or out, they are indeed conserved.
      The main interest of the Finite Volume method is that this property stays true through the spatial discretisation step.

      \paragraph{}
      The Finite Volume method consists in integrating the partial differential equation over each cell of the mesh.
      Writing $\mathcal{V}_i$ the volume of the $i$th cell:
      \begin{equation}
        \int_{\mathcal{V}_i} \frac{\partial \xi}{\partial t} \mathrm{d}v + \int_{\mathcal{V}_i} \nabla\cdot \vec{\operatorname{f}}\left(\xi\right) \mathrm{d}v = \int_{\mathcal{V}_i} \operatorname{S} \mathrm{d}v\ .
      \end{equation}
      Then the Green--Ostrogradski theorem transforms the flux divergence into a \PS{bilan surfacique}:
      \begin{equation}
        \frac{\mathrm{d}}{\mathrm{d} t} \int_{\mathcal{V}_i} \xi\mathrm{d}v + \oint_{\partial\mathcal{V}_i} \vec{\operatorname{f}}\left(\xi\right) \cdot \vec{\mathrm{d}s} = \int_{\mathcal{V}_i} \operatorname{S} \mathrm{d}v\ .
      \end{equation}
      By writing $\square_i = \frac{1}{\norm{\mathcal{V}_i}} \int_{\mathcal{V}_i} \square \mathrm{d}v$ the average in the $i$th cell, we then have:
      \begin{equation}
        \frac{\mathrm{d}\xi_i}{\mathrm{d} t}  + \frac{1}{\norm{\mathcal{V}_i}} \oint_{\partial\mathcal{V}_i} \vec{\operatorname{f}}\left(\xi\right) \cdot \vec{\mathrm{d}s} = \operatorname{S}_i \ .
      \end{equation}

      \paragraph{}
      As stated before, the spatial discretisation method does transform the partial differential equation into an ordinary differential equation.
      It tells us to store our quantities as the averaged values represented at the center of gravity of each cell as our vector $\Xi$.
      It also tells us to compute the divergence from equation (\ref{eq:pde_conservative}) as a \PS{bilan surfacique de flux}.
      The last thing to do is to decide how to compute this \PS{bilan surfacique de flux}.
      The cells from our meshes are polygons.
      Therefore they have a finite number of (planar) faces.
      The integral over the boundary of the cell can be decomposed by the faces, to get the approximation:
      \begin{equation}
        \oint_{\partial\mathcal{V}_i} \vec{\operatorname{f}}\left(\xi\right) \cdot \vec{\mathrm{d}s} \approx \sum_{j\textrm{ neighbor of } i} \vec{\operatorname{f}}_{ij} \cdot \vec{s_{ij}}
      \end{equation}
      where $\vec{\operatorname{f}}_{ij} \cdot \vec{s_{ij}}$ is an approximation of the flux going through the face between cells $i$ and $j$.
      This approximation is a key element of the Finite Volume method, therefore we will discuss it later.
      We can now compute the function $\operatorname{G}$ from equation (\ref{eq:ode}): for each face of the mesh we compute $\vec{\operatorname{f}}_{ij} \cdot \vec{s_{ij}}$, we add this value to the $i$th component and remove it from the $j$th component of our new vector.
      Then, after adding the source terms we get a vector containing the result of $\operatorname{G}\left(\Xi\right)$.
      As can be seen, every contribution of the flux added in a cell is removed from another, and therefore this spatial discretisation method preserves the \PS{conservativity} of the underlying equation.


    \subsection{The Riemann problem}

      \PS{Déplacer après la reconstruction ? C'est plus fondamental mais ça intervient après}

      \paragraph{}
      The last remaining problem with this presentation of the Finite Volume method is how to compute the flux going through cell interfaces.
      On the interfaces between two cells, we know the left and right quantities $\xi_L$ and $\xi_R$, and we need to compute the corresponding flux.
      It is possible here to use a reconstruction method to get a better approximation of the quantities left and right of the interface, and therefore we end up using the left and right quantities $\xi_L^*$ and $\xi_R^*$.
      The idea is now to compute the flux going through the face as a function of $\xi_L^*$, $\xi_R^*$ and the surface vector $\vec{s}$.
      From the interface point of view, there are two possible different states, one from each side: this is what is called a Rienamm problem.
      A Riemann problem is an initial value problem applied to a conservation equation, where the initial solution is piecewise constant with a single possible discontinuity.
      By working with the equation and deriving the jump condition, it is possible to compute the quantity at the interface from a possibly discontinuous state at the interface.
      Then it is possible to evaluate the flux associated with this state going through the surface.
      This approach can be called the exact Riemann solver as it uses the exact solution of the Riemann problem.
      But the drawback of this approach usually is the computational cost required to find this exact solution.
      What is usually done is to use approximate Riemann solvers, compromising between speed and accuracy.
      Several Riemann solvers\footnote{\PS{Est-ce qu'on parle toujours de solveur de Riemann (qui trouve la solution du problème idoine) ou on parle de "schéma de flux numérique" ?}} are available to the user in our solver, such as the well known Roe, HLLC or AUSM+ schemes \cite{Roe1981, Toro2009}.


    \subsection{Gradient reconstruction methods}

      \paragraph{}
      The standard Finite Volume method represents quantities with the averaged value in each cell.
      This corresponds to a first-order discretisation method.
      Simply put, it means that it can represent quantities exactly as 0-order polynomials locally to each cell.
      There are ways to achieve a higher order representation such as with the MUSCL approach \PS{(ref nécessaire ?)}.
      It consists in handling the surface flux evaluation on one hand, which is what we did in the Riemann problem part, and deciding what left and right quantities to feed to this flux computation on the other hand.
      In our solver, there are two ways to construct high-order states to give to the flux computation method.
      They are described in the following parts.
      For both of them, the idea is to use neighbouring data to enhance the order of the local representation.


      \subsubsection{The $k$-exact method}

        \paragraph{}
        The first method used to reconstruct high-order quantities is called the \emph{$k$-exact} method.
        The idea is to construct iteratively an order $k$ representation of the quantity using the neighbouring order $k-1$ representation \cite{HaiderCroisilleCourbet2009}.
        Usually, as do most of the solver users, this method is used to achieve a second-order reconstruction.
        But it can also achieve higher order reconstructions \cite{HaiderCroisilleCourbet2011, HaiderCourbetCroisille2018, PontBrennerCinellaEtAl2017}.
        At each step, while increasing the order of the representation, it is important not to create a local maximum or minimum.
        This might happen close to discontinuities in the solution, or near rapidly varying spots \PS{reformuler}.
        It is indeed common, when interpolating, to create local overshoot or undershoot.
        One might think here about Gibbs or Runge's phenomena, and despite the problem here being a different one, the idea is the same.
        Creating local extrema in the solution can be troublesome, and so the $k$-exact method limits the reconstructed polynomial to ensure it does not.


      \subsubsection{The Multislope method}

        \paragraph{}
        The second method used to reconstruct high-order quantities is called the \emph{Multislope} method.
        At each face, this method computes a local gradient to interpolate the quantity from the center of the cell to the face.
        This gradient is obtained using neighbouring data, with a complex mechanism that we will not discuss here \cite{LeTouzeMurroneGuillard2015}.
        Finally, the Multislope method gives a second-order reconstruction.
        Once again, this reconstruction might create local maxima or minima, and therefore it uses slope limiters \cite{Venkatakrishnan1993, BergerAftosmis2005} to prevent it.


      \paragraph{}
      We briefly explained above the spatial discretisation method used in our solver, as it might help our analysis of the time integration part.
      The Finite Volume method averages the partial differential equation over each cell of the mesh, which transforms the flux divergence into a \PS{bilan surfacique de flux}.
      A reconstruction method, the $k$-exact method or the Multislope method, is then used to get a higher-order representation of the solution so that the surface flux can be computed at each face.
      Slope limiters are used to prevent the formation of local extrema, which can be harmful to the computation.


  \section{Introduction to the time integration methods}

    \paragraph{}
    With the help of a spatial discretisation method, the equation we want to solve is now an ordinary differential equation.
    The main objective of this thesis is focused on the resolution of steady problems.
    The steady solution of equation (\ref{eq:ode}) is given by $\operatorname{G}\left(\Xi\right) = 0$.
    To get the solution, one might then try to find a root of the function $\operatorname{G}$.
    Unfortunately, with our typical applications, this function $\operatorname{G}$ has got bad mathematical properties, such as its stiffness, arising from the nonlinearities of the underlying equations.
    Therefore, algorithms that try to find a root of $\operatorname{G}$ struggle and usually fail.
    Another approach is to take an initial value $\Xi_0$, and to solve the equation (\ref{eq:ode}) for this initial value.
    After a long enough time, we hope that $\Xi$ will reach the desired steady solution.

    \paragraph{}
    The idea is now to solve the temporal evolution of $\Xi$ to get the solution after a long time when it approaches the steady solution.
    We will solve the equation numerically, which means we will iteratively compute the next solution after a given time step, knowing the current one.
    It is also possible to modify the equation, as we are interested in the final state, not in the transient one.
    It is possible for example to use local time stepping, which consists in having each cell of the mesh move forward in time with its own time step.
    \PS{C'est fou j'ai pas la moindre ref du pas de temps local de CEDRE, Lionel il n'y a rien sur le Upoint ?}
    The resulting transient states do not make sense from a physical point of view, as the equation solved is not the initial one, but it converges to the same steady solution.
    Therefore, it is alright to change the equation as long as it gives the same steady solution.
    Finally, this way of finding a converged steady solution is what is called a \emph{Pseudo-Transient Continuation} method \cite{KelleyKeyes1996}.

    \paragraph{}
    After deciding on an initial value, the equation we now want to solve is:
    \begin{equation}\label{eq:init_value_ode}
      \left\{\begin{aligned}
        & \frac{\mathrm{d} \Xi}{\mathrm{d}t} + \operatorname{G}\left(\Xi\left(t\right)\right) = 0 \\
        & \Xi\left(t_0\right) = \Xi_0
      \end{aligned}\right. \ .
    \end{equation}
    A time integration method is going to produce a succession of solutions: starting from $\Xi_0$, it produces $\Xi_1$ at $t_1$, then $\Xi_2$ at $t_2$, and $\Xi_n$ at $t_n$, etc.
    We will note $\Delta t_n$ the time step $t_{n+1} - t_n$.
    As we will mostly look at single steps of the time integration methods in the following, we will drop the subscript on the time step when it is not meaningful.
    We want to find a steady solution, and we are not interested in the evolution of the solution.
    It seems then reasonable to want to "go fast" to the steady state, meaning to use as large a time step as possible.
    Unfortunately, not every time integration method allows large time steps, as is well-known \cite{CourantFriedrichsLewy1967}.
    We need to define some tools to help us decide on the method we will use.


    \subsection{Analysis of time integration methods}

      \subsubsection{Consistency and order}

        \paragraph{}
        A time integration method must respect some properties to be "well-behaved".
        For instance, it has to be consistent.
        To define the consistency, we look at equation (\ref{eq:init_value_ode}).
        After one step, a numerical method gives a value $\Xi_1$, believed to be near the exact value $\Xi\left(t_0 + \Delta t\right)$.
        A numerical time integration method is said to be consistent if:
        \begin{equation}
          \lim_{\Delta t \rightarrow 0} \frac{\Xi_1 - \Xi\left(t_0 + \Delta t\right)}{\Delta t} = 0 \ .
        \end{equation}
        Also, the method is of order $p$ if the local error is in $\Delta t^{p+1}$ \cite{Iserles2008}:
        \begin{equation}
          \Xi_1 - \Xi\left(t_0 + \Delta t\right) = O\left(\Delta t^{p+1}\right) \ .
        \end{equation}
        This means a $p$ order method can recover exactly a solution that is a polynomial function of time of order less or equal to $p$.

        \paragraph{Note:}
        The order of a time integration method reflects its "local" behaviour, meaning on a single given time step, provided it is small enough.
        In the field of spatial discretisation of partial differential equations, the order $p$ of a method is such as:
        \begin{equation}
          \norm{\Xi - \Xi_{exact}} = O\left(h^p\right)
        \end{equation}
        where $h$ is the spatial discretisation parameter.
        We notice a difference between the two definitions: the error order of magnitude is $p+1$ for the temporal method and $p$ for the spatial one.
        This is due to the fact that the error we look at in the spatial case is global: it sums the error over the whole domain.
        It would correspond to counting the error on each step for the temporal integration.
        To convince oneself, we could say that when solving the differential equation \PS{on/in} an interval $\left[0, T\right]$ with a fixed $T$, the global error of a $p$ order method would \PS{be of, in, behave like ?} $O\left(\Delta t^p\right)$ as it amount to summing $T/\Delta t$ local errors of $O\left(\Delta t^{p+1}\right)$.
        We then get the coherency with the definition of the order for spatial discretisation methods.
        If this trick can help understand the difference between the two definitions, this is indeed just a mental trick and not a rigorous mathematical proof.
        To get this proof, more hypotheses on the method are required \cite{Iserles2008}.


      \subsubsection{Stability}

        \paragraph{}
        A meaningful criterion in the choice of a time integration method is stability.
        Depending on the application, we will expect different levels of stability in order to avoid a numerically induced divergence of the computation.

        \paragraph{}
        To analyse the stability of a time integration method, we usually apply it on the ordinary differential equation with a linear right-hand side: the Dahlquist test equation \cite{HairerWanner1996}.
        The reason is that if we have a solution $\tilde{\Xi}$ of equation (\ref{eq:init_value_ode}), we can linearise $\operatorname{G}$ in $\tilde{\Xi}$.
        With $y = \Xi - \tilde{\Xi}$ and $J = -\frac{\partial \operatorname{G}}{\partial \Xi}\left(\tilde{\Xi}\right)$, assumed constant, we then have:
        \begin{equation}\label{eq:dahlquist}
          \frac{\mathrm{d} y}{\mathrm{d} t} = J y \ .
        \end{equation}
        This new equation used to analyse the stability of time integration methods is the one called the Dahlquist test equation.
        We look at this equation in $\mathbb{C}$, so that we can compute the eigenvalues and eigenvectors of the matrix $J$.

        \paragraph{Note:}
        When looking at a method applied to the Dahlquist test equation (\ref{eq:dahlquist}), we assume that the real part of the eigenvalues of $J$ are all negative.
        This choice may seem arbitrary but can be understood with the following example.
        Let us work in $\mathbb{C}^2$, with:
        \begin{equation}
          J = \begin{pmatrix} -1 & 0 \\ 0 & 10^3 \end{pmatrix}, \quad y_0 = \begin{pmatrix} 1 \\ 0 \end{pmatrix} \ .
        \end{equation}
        The solution of the equation is then:
        \begin{equation}
          y\left(t\right) = \begin{pmatrix} e^{-t} \\ 0 \end{pmatrix} \ .
        \end{equation}
        As we solve the equation numerically, the floating point representation introduces some roundoff error.
        The initial condition may then be
        \begin{equation}
          y_0' = \begin{pmatrix} 1 \\ \epsilon \end{pmatrix}
        \end{equation}
        instead of the exact one $y_0$, with a typical $\epsilon = 10^{-15}$ for double precision.
        Let us suppose that we have an exact time integration method that gives the exact solution at each time step $t_n = n\Delta t$.
        The solution computed by this method will be:
        \begin{equation}
          y_n = \begin{pmatrix} e^{-n\Delta t} \\ \epsilon e^{10^3 n \Delta t} \end{pmatrix}
        \end{equation}
        that gives an error of $\epsilon e^{10^3 n\Delta t}$.
        For the numerical values suggested here, this amount to an error as large as $10^6$ for $n = 5$ and $10^{28}$ for $n = 10$.
        The explosion of the error comes from the fact that the positive eigenvalue of $J$ amplifies the roundoff error.
        This phenomenon has nothing to do with the time integration scheme, but with the equation.
        Finally, this is why we study equation (\ref{eq:dahlquist}) assuming the eigenvalues of $J$ are negative.


        \paragraph{Single step methods}
        To compute the solution at the next time step, some methods need only to know the current solution.
        Such methods are called single steps methods.
        When applied to the Dahlquist test equation (\ref{eq:dahlquist}), we write for a single-step method:
        \begin{equation}\label{eq:single_step}
          y_{n+1} = g\left(\Delta tJ\right)y_n
        \end{equation}
        the relationship between the current solution and the next one.
        For most time integration methods, $g$ is an analytic function.
        By decomposing the initial value on a basis of eignevectors of $J$, $v_1, \dots, v_N$, associated to the eigenvalues $\alpha_1, \dots, \alpha_N$, we write: $y_0 = \sum_{i=1}^N \lambda_i v_i$.
        Because $g$ is an analytic function, we then have:
        \begin{equation}
          y_n = \sum_{i=1}^N \lambda_i g\left(\Delta t \alpha_i\right)^n v_i \ .
        \end{equation}

        It is now straightforward to deduce a stability condition for the single step method: if for any $i$ we have $\left|g\left(\Delta t\alpha_i\right)\right| < 1$, then $y_n$ converges to 0.
        This is how we can define the stability region of a time integration method:
        \begin{equation}
          \left\{ \, z \in \mathbb{C} \; \mid \; \left| g\left(z\right) \right| < 1 \, \right\} \ .
        \end{equation}
        When each eigenvalue of $J$ falls in the stability region, then the method is stable.

        \paragraph{}
        If an eigenvalue is not in the stability region, the associated \PS{direction propre} will be amplified and a numerical instability will lead to the divergence of the computation.
        We can see that the argument of the function $g$ is not $J$ but $\Delta t J$.
        This means that stability can be achieved by choosing wisely the time step: with a small enough $\Delta t$ we can make sure that each eigenvalue falls into the stability region.
        Unfortunately, this often forces the user to set a relatively small time step, which is an issue for our steady computations.


        \paragraph{Multi step methods}
        Some methods do not fall into the previous framework.
        The multi step methods, in particular, cannot be written under the form of equation (\ref{eq:single_step}).
        These methods use not only $y_n$ to find $y_{n+1}$, but the $k$ previous steps.
        Applied to equation (\ref{eq:dahlquist}), they can be written in the form:
        \begin{equation}
          y_{n+1} = \sum_{i=1}^k g_{k-i}\left(\Delta t J\right) y_{n+1-i} \ .
        \end{equation}
        When we look for $y_i$ under the form $y_i \propto \mu^{i}$, we then have:
        \begin{equation}
          \mu^k = \sum_{i=1}^k g_{k-i}\left(\Delta t J\right) \mu^{k-i}
        \end{equation}
        which leads us to identifying the polynomial $g_{\Delta t J}\left(\mu\right) = \mu^k - \sum_{i=0}^{k-1}g_i\left(\Delta t J\right)\mu^i$.
        If each root of this polynomial is of modulus less than 1, the solution converges to 0.
        This is how we can define the stability region for multi step methods \cite{HairerWanner1996}:
        \begin{equation}
          \left\{ \, z \in \mathbb{C} \; \mid \; \parbox{25em}{all roots of $X^k - \sum_{i=0}^{k-1}g_i\left(z\right)X^i$ are of modulus less or equal to 1, strictly less to 1 for roots with multiplicity}
           \, \right \} \ .
        \end{equation}

        \paragraph{}
        The key property resulting in the stability analysis of time integration methods is the \emph{A-stability} \cite{Dahlquist1963}.
        A time integration method is A-stable if its stability region contains the left half complex plane.
        Simply put, a method is A-stable if it converges to 0 when it should, and does not diverge due to numerical errors.
        The A-stability is interesting to us, as an A-stable method is also said to be unconditionally stable, whereas a method that is not is conditionally stable.
        This other characterisation comes from the fact that a non-A-stable method needs to respect some additional criteria to be stable, on the time step it uses for example, whereas an A-stable method is stable no matter the time step.
        As we said before, we would like to use large time steps to quickly find the steady state of our applications, and that is why  we look for A-stability in our methods.


    \subsection{Explicit methods}

      \paragraph{}
      Explicit methods are called this way because, at each step, the computation of the next solution is straightforward: they give it explicitly as a function of currently available data.
      They are largely used in unsteady computational fluid dynamics simulations.
      Their strength comes from the fact that they are usually simple and therefore easy to implement in a solver, and computationally inexpensive compared to non-explicit methods.


      \subsubsection{Explicit Euler method}

        \paragraph{}
        The explicit Euler method is the most simple time integration method.
        It consists in integrating equation (\ref{eq:init_value_ode}) between $t_n$ and $t_{n+1}$ assuming the function $\operatorname{G}$ stays constant, equal to $\operatorname{G}\left(\Xi_n\right)$.
        Equivalently, it consists in replacing the time derivative $\frac{\mathrm{d} \Xi}{\mathrm{d} t}$ by a finite difference $\frac{\Xi_{n+1} - \Xi_n}{\Delta t}$ and evaluating $\operatorname{G}$ in $\Xi_n$.
        Then, the method gives:
        \begin{equation}
          \Xi_{n+1} = \Xi_n - \Delta t \operatorname{G}\left(\Xi_n\right) \ .
        \end{equation}

        \paragraph{}
        After verifying that this is a first-order method, a quick stability analysis gives a stability region equal to the open unity disk centred in -1.
        Practically, this stability region is often deemed unsatisfactory as it forces the use of small time steps.
        Yet this method is a classic that we had to introduce before talking about more complex methods.


      \subsubsection{Runge--Kutta methods}

        \paragraph{}
        Instead of making one step forward in time as the explicit Euler method, Runge--Kutta methods will make a set of intermediate steps, and the final solution is found as a combination of those intermediate steps.
        The general idea is as such.
        Supposing we know the value of $\Xi_n$ in $t_n$, we set the intermediates steps $t_{n, i} = t_n + c_i\Delta t$ for $1 \leq i \leq k$, with a fixed $k$.
        We can now integrate exactly between $t_n$ and $t_{n, i}$ equation (\ref{eq:init_value_ode}):
        \begin{equation}
          \Xi\left(t_{n, i}\right) = \Xi_n - \Delta t \int_{t_n}^{t_{n,i}} \operatorname{G}\left(\Xi\left(t\right)\right) \mathrm{d}t\ .\
        \end{equation}
        The integral on the right-hand side is then approximated with a \PS{quadrature} using the previously computed intermediate steps:
        \begin{equation}
          \int_{t_n}^{t{n,i}} \operatorname{G}\left(\Xi\left(t\right)\right) \mathrm{d}t \approx \sum_{j = 1}^{i-1} a_{ij} \operatorname{G}\left(\Xi\left(t_{n,j}\right)\right) \ .
        \end{equation}
        Once each intermediate step in known, we finally integrate between $t_n$ and $t_{n+1}$ equation (\ref{eq:init_value_ode}) and approximate the integral using the intermediate steps.

        \paragraph{}
        To sum up, a Runge--Kutta method iterates in the following way:
        \begin{equation}
          \left\{\begin{aligned}
            \Xi_{n+1} &= \Xi_n - \Delta t \sum_{i = 1}^k b_i \operatorname{G}\left(\Xi_{n,i}\right) \\
            \textrm{with}\quad \Xi_{n,i} &= \Xi_n - \Delta t \sum_{j = 1}^{i-1} a_{ij} \operatorname{G}\left(\Xi_{n,j}\right)
          \end{aligned}\right. \ .
        \end{equation}

        \paragraph{}
        A Runge--Kutta method is characterised by its size $k$ and by the quadrature coefficients: $a_{ij, 1\leq j<i\leq k}$, $b_{i, 1\leq i\leq k}$ and $c_{i, 1\leq i\leq k}$.
        There are as many Runge--Kutta methods as there are choices in the quadrature coefficients, but not all choices give good methods.
        There are criteria that the coefficients must follow to ensure the consistency of the method, and then criteria with more complexity as the order increases.
        The quadrature coefficients are often arranged in the Butcher tableau:
        \begin{equation}
          \begin{array}{c|c}
            c & A \rule[-1.1ex]{0pt}{0pt} \RKBar \transpose{b}
          \end{array}
          \qquad = \qquad
          \begin{array}{c|ccccc}
            0\\
            c_2    & a_{21} \\
            c_3    & a_{31} & a_{32} \\
            \vdots & \vdots &        & \ddots\\
            c_k    & a_{k1} & a_{k2} & \hdots & a_{k,k-1} \RKBar
            b_1    & b_2    & \hdots & b_{k-1} & b_k
          \end{array} \ .
        \end{equation}

        \begin{table}
          \begin{tabular}{P{.15\textwidth}P{.3\textwidth}P{.4\textwidth}}
            \begin{tabular}{c|c}
              0 \RKBar 1
            \end{tabular} &
            \begin{tabular}{c|cc}
              0 \\ 1/2 & 1/2 \RKBar 0 & 1
            \end{tabular} &
            \begin{tabular}{c|cccc}
              0 \\ 1/2 & 1/2 \\ 1/2 & 0 & 1/2 \\ 1 & 0 & 0 & 1 \RKBar 1/6 & 1/3 & 1/3 & 1/6
            \end{tabular} \\
            RK1 & RK2 & RK4 \\
          \end{tabular}
          \caption{Butcher tableau for the explicit Euler, Midpoint and RK4 methods.}\label{tab:rk_butcher}
        \end{table}

        \paragraph{}
        The Butcher tableau of some well-known Runge--Kutta methods are shown in table \ref{tab:rk_butcher}.
        The RK1 method is in fact equivalent to the explicit Euler method.
        The RK2 method is one of the 2 steps second-order Runge--Kutta method.
        This one is also called the Midpoint method.
        The RK4 method is a fourth-order method with 4 steps.
        This is the most famous Runge--Kutta method, vastly used for explicit time integration of ordinary differential equations.

        \paragraph{}
        It can be shown that the order $p$ of the method is less or equal to the number of steps $k$.
        Up to 4 steps, it is possible to choose the quadrature coefficients to have $p = k$.
        Above that, getting a bound on the order depending on the number of steps is still an open problem as of today.
        For a Runge--Kutta method of order $p$, the corresponding function used for the stability analysis is \cite{HairerWanner1996}:
        \begin{equation}
          g\left(z\right) = 1 + z + \frac{z^2}{2} + \dots + \frac{z^p}{p!} + O\left(z^{p+1}\right) \ .
        \end{equation}
        When $p = k$, the last term $O\left(z^{p+1}\right)$ is in fact null.
        We show in figure \ref{fig:rk_stab} the stability region of the Runge--Kutta methods of orders up to 4.

        \begin{figure}
          \centering
          \includegraphics[width=.45\textwidth]{figures/rk_stab.png}
          \caption{Stability regions (in colour) of the four first-order Runge--Kutta methods.}
          \label{fig:rk_stab}
        \end{figure}

        \paragraph{}
        We can see that Runge--Kutta methods are not A-stable, and they do not have a large stability region.
        Increasing the order of the method does increase the stability region, but not enough for our applications.
        Practically, this imposes the use of small time steps, which is agreeable for unsteady computations but not for our steady ones.
        If one iteration of the method is inexpensive, the total number of iterations needed to reach the steady solution will make the overall computation too costly.


      \subsubsection{Adams--Bashforth methods}

        \paragraph{}
        We could try to use other explicit methods, such as multi-step Adams--Bashforth methods.
        The $k$th order Adams--Bashforth method uses the last $k$ computed steps to find the next one.
        One can indeed check that the index $k$ designating the method also corresponds to its order \cite{HairerNorsettWanner1993}.

        \paragraph{}
        The idea is to apply a Lagrange interpolation of the function $\operatorname{G}$ from equation (\ref{eq:init_value_ode}) in the $k$ last computed points, and then replace $\operatorname{G}$ with the interpolation polynomial when integrating from $t_n$ to $t_{n+1}$.
        Contrary to the Runge--Kutta methods, as we reuse previous information, a single $\operatorname{G}$ evaluation is required at each step.
        The cost of one iteration of the Adams--Bashforth method is then really cheap \PS{(reformuler)}.
        The stability analysis for such methods is a bit more complex \cite{HairerNorsettWanner1993, HairerWanner1996}, and so we show the result that we obtained numerically on figure \ref{fig:ab_stab} without \PS{faire le calcul : deriving the calculus ?}.
        The conclusion is even worse than with Runge--Kutta methods, as the stability region decreases as the order increases.
        The Adams--Bashforth can reach a high order of accuracy while staying computationally inexpensive, but they lack drastically in stability, and that is why they are often not used in computational fluid dynamics computations.
        More generally, an explicit multi-step method cannot be A-stable \cite{Dahlquist1963}.

        \begin{figure}
          \centering
          \includegraphics[width=.6\textwidth]{figures/ab_stab.png}
          \caption{Stability regions (in colour) of the four first Adams--Bashforth methods.}
          \label{fig:ab_stab}
        \end{figure}


    \subsection{Implicit methods}

      \paragraph{}
      We explained in the previous section why explicit time integration methods are not suited for our applications.
      It is then natural to look at implicit methods.
      Contrary to the explicit methods, implicit methods do not give the looked-for solution right away, but as the solution of a specific equation.
      The name is appropriate: the next state is not given explicitly but implicitly.


      \subsubsection{Implicit Euler method}

        \paragraph{}
        The implicit Euler method is the equivalent of the explicit Euler method but on the implicit side.
        It is quite similar, as it consists in integrating equation (\ref{eq:init_value_ode}) assuming the function $\operatorname{G}$ is constant, but this time equal to $\operatorname{G}\left(\Xi_{n+1}\right)$.
        We then have:
        \begin{equation}
          \Xi_{n+1} = \Xi_n - \Delta t \operatorname{G}\left(\Xi_{n+1}\right)
          \Leftrightarrow \Xi_{n+1} - \Xi_n + \Delta t \operatorname{G}\left(\Xi_{n+1}\right) = 0
        \end{equation}
        and the next state $\Xi_{n+1}$ is given as the solution of a nonlinear problem, the root of a nonlinear function.

        \paragraph{}
        After checking that this is a first-order method, we can also do the stability analysis to find the corresponding function $g\left(z\right) = \left(1 - z\right)^{-1}$, that gives a stability region equal to the whole complex plane minus the closed unity disk centered in 1.
        Therefore this method is A-stable.


      \subsubsection{Implicit Runge--Kutta methods}

        \paragraph{}
        Runge--Kutta methods can also be implicit methods.
        This happens when the quadratures use points that have not already been computed.
        In other words, this corresponds to a full $A$ matrix in the butcher tableau, where it is strictly lower triangular for explicit Runge--Kutta methods.
        This also means that any step of the Runge--Kutta method may be used in any other step, and therefore we may have to simultaneously solve an implicit system of equations.
        This may lead to awfully expensive methods, and therefore users tend to restrict themselves to some particular methods.

        \paragraph{}
        Despite their cost, implicit Runge--Kutta methods can be appealing.
        The main reason is that they can easily achieve a high order of accuracy.
        For example, the methods based on Gauss--Legendre quadratures achieve an order $2k$ with $k$ steps, and they are all A-stable \cite{Iserles2008}.
        Theoretically, this means we can achieve an arbitrarily high order while keeping the stability quality with these methods.
        However, users usually stop at the 3-stage 6th order method, as the computational cost tends to be too much for higher order methods.

        \paragraph{}
        When applying the stability analysis to a Runge--Kutta method, we can derive the function $g$ using the arrays $A$, $b$ and $c$:
        \begin{equation}
          g\left(z\right) = 1 + z \transpose{b} \left(\operatorname{Id} - zA\right)^{-1} \transpose{\left(1, \dots, 1\right)} \ .
        \end{equation}
        We will not try to show the corresponding stability regions as there are too many methods possible.
        Instead, we will review some of the most frequent from the literature.

        \paragraph{Diagonally Implicit Runge--Kutta methods}
        The Diagonally Implicit Runge--Kutta methods \cite{Alexander1977}, or DIRK methods, are Runge--Kutta methods with a lower triangular matrix $A$.
        Then, each step is given as an implicit problem using the already known steps and the next one.
        The difference is that instead of solving a full implicit system, we just need to solve each implicit step successively.
        This help drastically reduces the cost of the method.
        Furthermore, if all quadrature coefficient $a_{ii}$ are equals, as we invert at each step the matrix $\operatorname{Id} + a_{ii} \Delta t \frac{\mathrm{d} \operatorname{G}}{\mathrm{d} \Xi}\left(\Xi_n\right)$, this can help the linear solve.
        This variant is called Singly Diagonally Implicit Runge--Kutta methods (SDIRK) \cite{HairerWanner1996}.

        \paragraph{Rosenbrock methods}
        Rosenbrock methods are also called linearly implicit Runge--Kutta methods.
        \PS{TODO ?}


      \subsubsection{Backward differentiation formula}

        \paragraph{}
        As we extended the explicit Runge--Kutta methods to implicit methods, we can also extend the Adams--Bashforth method to an implicit one.
        When interpolating the function $\operatorname{G}$, we add one point to the Lagrange interpolation: the point we want to compute.
        This new method is called the Adams--Moulton method.
        However, the stability region of Adams--Moulton is quite narrow, as they were originally not made for stiff equations \cite{Iserles2008}.
        This is why the BDF methods were introduced.
        Contrary to Adams--Bashforth and Adams--Moulton methods, they use the Lagrange interpolation of the solution $\Xi$ instead of the function $\operatorname{G}$.
        Then, we can replace the time derivative of the solution with the time derivative of the interpolating polynomial, and evaluate this new equality at the next time step $t_{n+1}$.
        This gives an implicit equation we need to solve to get $\Xi_{n+1}$.
        The name of those methods come from the fact that if we use a constant time step between iterations, this equation can be written using the differentiating operator defined by $\nabla^0 \square_i = \square_i$ and $\nabla^{j+1} \square_i = \nabla^j \square_i - \nabla^j \square_{i-1}$:
        \begin{equation}
          \sum_{i=1}^k \frac{1}{i} \nabla^i \Xi_{n+1} + \Delta t \operatorname{G}\left(\Xi_{n+1}\right) = 0
        \end{equation}

        \paragraph{}
        We can show that the order of the method is equal to the index of the method, corresponding to the number of previous states needed to compute the next one.
        These methods allow for an arbitrarily high order without increasing the cost, as the implicit equation is not harder to solve as the order increases.
        However, the stability analysis limits the higher order achievable.
        The stability analysis of the BDF methods can be done numerically.
        At this stage, we can note that the first-order BDF method is in fact the implicit Euler method.
        Also, methods of order 7 or higher are unstable, so we can limit our analysis to methods with orders from 1 to 6.
        The corresponding stability regions are satisfying, as can be seen in figure \ref{fig:bdf_stab}.
        Particularly, the first and second methods are A-stable.
        More generally, there are no A-stable multi steps methods with an order higher than 2 \cite{Dahlquist1963, HairerWanner1996}.

        \begin{figure}
          \centering
          \includegraphics[width=\textwidth]{figures/bdf_stab.png}
          \caption{Stability regions (in colour) of the BDF methods.}
          \label{fig:bdf_stab}
        \end{figure}

      \paragraph{}
      This introduction to the classic time integration methods helps us to decide what to do for our solver.
      As we said earlier, we want to solve an ordinary differential equation in order to recover the steady solution, reached after a long time.
      Then, explicit methods that constrain the time step are not well fitted.
      Even if they cost more, in an algorithmic way, implicit methods are indeed the best choice for our stiff equations.
      It is often better to do a single expensive iteration over a large time step with an implicit method, than making a lot of inexpensive iterations over small time steps with an explicit method.
      This is finally why we will continue working with the A-stable implicit Euler method.
      This method is already used in our solver as the base implicit method, for the same reasons we want to use it.
      Furthermore, it is at the base of all other implicit methods, as they can be seen as small variations of the implicit Euler methods.
      Choosing it is also smart as updating it into another implicit time integration method will not require too much work.


  \section{Implicit methods framework}
  \PS{Revoir structure}


    \subsection{Methodology of the implicit time integration}

      \paragraph{}
      As we explained in the previous section, implicit time integration methods give the next state as the solution of a nonlinear equation or a system of nonlinear equations.
      If we want to use implicit methods, we then need to be able to solve a nonlinear problem.
      We will continue our discussion using the implicit Euler method, but everything can easily be adapted for any other implicit method.
      For Diagonally Implicit Runge--Kutta methods, we apply the nonlinear solve successively for each step.
      For BDF methods, we apply the nonlinear solve on a slightly different equation.


      \subsubsection{From a nonlinear problem \dots}

      	\paragraph{}
      	We can use Newton's method to solve a nonlinear problem of the form:
      	\begin{equation}
      		\tilde{f}\left(\Xi_{n+1}\right) = 0 \ .
      	\end{equation}
        Equivalently, as we work here for a single $n$ at a time, we could express this equation in terms of the increment $x = \Xi_{n+1} - \Xi_n$:
        \begin{equation}\label{eq:nonlinear}
      		f\left(x\right) = 0 \ .
      	\end{equation}
        Newton's method starts from an initial guess $x_0$.
        We usually take $x_0 = 0$ as it equivalent to take $\Xi_n$ as an initial guess for $\Xi_{n+1}$.
        The method will then iterate to approximate the solution of equation (\ref{eq:nonlinear}).

        \paragraph{}
        At each step of Newton's method, we linearise \PS{au premier ordre} in the current estimation $x_i$ the nonlinear function $f$ evaluated  in the next iteration $x_{i+1}$ and we set this linearisation equal to 0, the desired value:
        \begin{equation}\label{eq:nonlinear_linearised}
          f\left(x_{i+1}\right) \approx f\left(x_i\right) + f'\left(x_i\right) \left( x_{i+1} - x_i \right) = 0
        \end{equation}
        This gives a linear problem in which $x_{i+1}$ is the solution.
        \PS{Refs sur méthode de Newton ?}


      \subsubsection{\dots to a linear one}

        \paragraph{}
        We can rewrite equation (\ref{eq:nonlinear_linearised}) into the classic linear problem \PS{notation x déjà utilisée}:
        \begin{equation}\label{eq:linear}
          Ax = b
        \end{equation}
        where we can identify $A = f'\left(x_i\right)$, $b = -f\left(x_i\right)$ and $x = \left( x_{i+1} - x_i \right)$.
        A lot of methods were conceived to solve such linear problems, and we will discuss later how we will handle them.


      \paragraph{}
      To sum up, we started from the partial differential equation (\ref{eq:pde}) arising from the physical model.
      With a spatial discretisation method, and after choosing an initial value, this transforms into an ordinary differential equation (\ref{eq:init_value_ode}).
      For stability reasons, we decided to use implicit time integration methods.
      Iteratively, such methods are going to produce one or several nonlinear problems in the form of equation (\ref{eq:nonlinear}).
      Newton's method used to solve such problems is going to produce a succession of linear problems in the form of equation (\ref{eq:linear}).
      This complicated sequence of operations describes the time integration procedure used in our solver to find the solution to steady problems.
      It is schematised in figure \ref{fig:steady_solve}, where the blue circles correspond to the problems being solved, while green circles correspond to the methods used to solve them.
      In this thesis, we are interested in the non-greyed parts.

      \begin{figure}
        \centering
        \includegraphics[width=.8\textwidth]{figures/steady_solve.png}
        \caption{Procedure used to find the solution of steady problems.}
        \label{fig:steady_solve}
      \end{figure}


    \subsection{Nonlinear solver}

      \paragraph{}
      As we said, we solve the nonlinear problem with Newton's method.
      What was done in our solver was in fact a single step of Newton's method, which means a single linearization of the nonlinear problem.
      We thought we would benefit from an actual Newton's method, so we implemented it.
      Contrary to what was done before, several linear problems need to be solved during one single time step, which means using several Jacobian matrices.

      \paragraph{}
      The issue when using Newton's method is that it should be accompanied by a Line Search algorithm to determine the step size $\alpha$ so that the next iterate of the method is:
      \begin{equation}
        x_{i+1} = x_i - \alpha f'\left(x_i\right)^{-1} f\left(x_i\right) \ .
      \end{equation}
      During this thesis, we did some work towards using a complete Newton's method, but we did not have time to develop a Line Search algorithm, so we ended up using the standard method from our solver: a single linearization of the nonlinear problem.


    \subsection{Linear solver}

      \paragraph{}
      We want to be able to solve efficiently linear systems like (\ref{eq:linear}).
      We additionally assume that $A$ is an invertible matrix.
      Let us note the size of the linear system $N$.
      This size is quite large in our typical applications, but the linear system is sparse.
      This means that the coefficients of the matrix $A$ are mostly zeros.
      This is because a coefficient in the matrix $A$ corresponds to a link between two degrees of freedom.
      For the spatial discretisation methods we use, a cell depends on a small number of neighbouring cells, and therefore a degree of freedom is not linked with most of the others \PS{(c'est clair ?)}.
      Such sparse matrices can be seen in figure \ref{fig:sparse}.
      Using the sparsity of the matrix is essential, as it would not be possible to store it as a dense matrix in the memory of a computer.
      Instead, we use some clever formats such as the Compressed Sparse Row format \cite{Saad2003}.
      Some operations such as matrix-vector products are more efficiently done using such formats.

  		\begin{figure}
  			\centering
  			\begin{subfigure}[t]{0.3\textwidth}
  				\centering
  				\includegraphics[width=\textwidth]{figures/GT01R.png}
          \caption{GT01R: 2D inviscid flow in the inter-blade channel of a linear cascade turbine.}
  				\label{fig:sparse.GT01R}
  			\end{subfigure}
  			\hfill
  			\begin{subfigure}[t]{0.3\textwidth}
  				\centering
  				\includegraphics[width=\textwidth]{figures/HV15R.png}
  				\caption{HV15R: 3D RANS simulation of an engine fan.}
  				\label{fig:sparse.HV15R}
  			\end{subfigure}
  			\hfill
  			\begin{subfigure}[t]{0.3\textwidth}
  				\centering
  				\includegraphics[width=\textwidth]{figures/RM07R.png}
  				\caption{RM07R: 3D viscous flow in a jet engine compressor.}
  				\label{fig:sparse.RM07R}
  			\end{subfigure}
  			\caption{Matrices from various CFD simulations \cite{PacullAubertBuisson2011} using a Finite Volume method. Coloured points correspond to nonzero values.}
  			\label{fig:sparse}
  		\end{figure}

      \paragraph{}
      Many methods exist to solve linear problems.
      Some are even taught in school, such as the Gaussian elimination.
      Such methods are called direct methods, as they first do some work and then find directly the exact solution to the problem.
      But for problems with huge sizes such as the ones we encounter in computational fluid dynamics, the amount of work is too much to be computed with today's means.
      It would take too much time as well as too much memory.
      Instead, we can use iterative methods.
      An iterative method starts from an initial guess $x_0$ and produces as it iterates a supposedly better estimation of the solution $x_n$.
      The subscript $n$ has nothing to do with the subscript used to identify the iterates of the time integration method: we work at a given fixed time step here.
      We can then decide when to stop the method, whether the solution estimate is good enough or the resolution is taking too much time.
      Iterative methods are the most well-fitted to solve large sparse linear problems, and they are the preferred solution in computational fluid dynamics.

      \paragraph{}
      Among the iterative methods are what could be called the "classic" methods, or relaxation.
      They are the Jacobi, Gauss--Seidel or Successive Over Relaxation methods.
      They decompose the matrix into $A = M - N$ with $M$ a matrix easily invertible.
      The choice of $M$ depends on the method.
      Then, starting from a given $x_0$, at each step we compute $x_{n+1} = M^{-1} \left( N x_n + b \right)$.
      Those methods are often deemed not efficient enough for computational fluid dynamics applications.
      They are often used, however, as preconditioning methods, as we will see later.

      \paragraph{}
      Another class of iterative methods is becoming the standard for computational fluid dynamics: the Krylov subspace methods.
      A Krylov subspace method projects the linear problem (\ref{eq:linear}) on a smaller linear subspace called Krylov subspace.
      The obtained smaller system is then solved, much more easily.
      The cleverness resides in the fact that those Krylov subspaces are \PS{emboités}, and each iteration reuses the information obtained in the previous ones.

      \paragraph{}
      In the following, we note at iteration $n$ the residual $r_n = b - A x_n$.
      Practically, we keep $n \ll N$ so that the cost of the method stays reasonable, but this does not change the following.
      The corresponding Krylov subspaces are defined as:
  		\begin{equation}
  			\krylov[A, r_0]{n} = \operatorname{Vect}\left( r_0, A r_0, \dots, A^{n-1} r_0 \right) \ .
  		\end{equation}
      Then, we seek the next iterate in $\krylov{n}$ satisfying a Petrov--Galerkin condition \cite{SimonciniSzyld2007}:
  		\begin{equation}
  			x_n \in x_0 + \krylov[A, r_0]{n} \quad \textrm{such as} \quad r_n \perp \mathcal{L}_n
  		\end{equation}
      where $\mathcal{L}_n$ is a linear subspace with dimension $n$.
      For example, $\mathcal{L}_n = \krylov{n}$ corresponds to a Galerkin condition, and $\mathcal{L}_n = A \krylov{n}$ is a minimum residual condition.

      \paragraph{}
      To construct the growing Krylov subspace, we can use the Arnoldi iteration \cite{TrefethenBau1997}.
      As a result, the matrix $A$ is only used through matrix-vector products.
      Indeed, Krylov subspace methods do not require the matrix $A$ to solve the linear system (\ref{eq:linear}), just to know how to compute matrix-vector products.
      We will use this property later on.

      \paragraph{}
      There are many Krylov subspace methods that can solve a linear problem, characterised by the choice of $\mathcal{L}_n$.
      The minimal residual condition gives the Generalized Minimal Residual method or GMRES \cite{SaadSchultz1986}, vastly used in computational fluid dynamics \cite{FrancoCamierAndrejEtAl2020} and other fields of numerical simulations \cite{ErnstGander2012, Mercier2015}.
      Its main advantage is that even if its iterations may be slightly more expensive than other methods such as the Bi-CGSTAB method \cite{Vorst1992, TrefethenBau1997}, the residual norm is minimised and the error is then decreasing as the method iterates \PS{phrase pas claire ?}.
      This allows for some control of the residual norm.
      With the Bi-CGSTAB method for example, also available in our solver, the residual norms do not have to be decreasing, which can lead to chaotic convergence.
      The GMRES method already exists in our solver and is often used.
      Because it is discussed a lot in the literature, many variants and enhancements were developed \cite{CoulaudGiraudRametEtAl2013, Vasseur2016, JolivetTournier2016}.
      For those reasons, we decided to keep the GMRES method as the base of our linear solver.

      \paragraph{}
      The convergence of Krylov subspace methods, and more generally of iterative methods, depends on the linear system matrix.
      It is common knowledge that the convergence is linked to the condition number of the matrix \cite{Nevanlinna1994}.
      The condition number of an invertible matrix $A$ is defined as:
      \begin{equation}
        \kappa\left( A \right) = \norm{A} \norm{A^{-1}}
      \end{equation}
      and so it depends on the chosen norm.
      A problem is said to be well conditioned when $\kappa = O\left(1\right)$, and ill-conditioned when $\kappa \gg 1$.
      For the Euclidean norm $\norm[2]{\cdot}$, we have
  		\begin{equation}\label{eq:conditioning}
  			\kappa\left( A \right) = \frac{\sigma_{\max}}{\sigma_{\min}}
  		\end{equation}
      where $\sigma_{\min}$ and $\sigma_{\max}$ are the smallest and largest singular values of the matrix $A$.
      As a reminder, the singular values of $A$ are the eigenvalues of $A^* A$ where $A^*$ notes the conjugate transpose of $A$.
      The relation (\ref{eq:conditioning}) is often simplified as
      \begin{equation}
        \kappa\left( A \right) = \frac{\left|\lambda_{\max}\right|}{\left|\lambda_{\min}\right|}
      \end{equation}
      with $\lambda_{\min}$ and $\lambda_{\max}$ the eigenvalue of $A$ with the smallest and largest modulus.
      If this helps picture what the condition number stands for, this is only true for normal matrices: matrices that commute with their conjugate transpose.
      However, the matrices we face in our field have no reasons to be normal matrices.
      An amusing result is found in \cite{GreenbaumPtakStrakos1996}.
      For a given decreasing convergence curve and a given spectrum, it is possible to construct a linear problem such as the convergence of GMRES follows the given curve and the matrix has the given spectrum.
      In other words, there are ill-conditioned matrices for which GMRES converges quickly and well-conditioned matrices for which GMRES will be slow.
      In particular, we can find a matrix with the best condition number possible, meaning 1, on which GMRES will make no progress until the last iteration.
      This is because of the nonnormality of the matrix \cite{GreenbaumStrakos1994, GreenbaumPtakStrakos1996}.
      To handle it in the analysis of the convergence, one must not look at the spectrum but the pseudospectrum \cite{Trefethen1999}.
      This analysis is quite complex and is mostly done in an analytical context, not an industrial one.
      Using Krylov subspace methods with nonnormal matrices and the study of their convergence is no simple task \cite{LiesenTichy2004, Huhtanen2005}.
      As the matrices we encounter in our computational fluid dynamics problems are \PS{quelconques}, meaning \PS{à priori non-normales}, we decided not to study finely the spectrum of the operator.

      \paragraph{}
      Even if we will not look deeply into the spectrum of the matrix, we still help the linear solver with preconditioning.
      Preconditioning consists in multiplying the equation (\ref{eq:linear}) with a preconditioning operator $P$ to the left for left preconditioning:
      \begin{equation}
        PAx = Pb
      \end{equation}
      and to the right for right preconditioning:
      \begin{equation}
        \left\{\begin{aligned} APx' &= b \\	Px' &= x \end{aligned}\right. \ .
      \end{equation}
      The idea is to transform $A$ into a matrix that is more easily invertible.
      The matrix we need to invert is now $PA$ for left preconditioning and $AP$ for right preconditioning.
      If the preconditioning operator is close to $A^{-1}$ and is cheap to compute, this new matrix is close to $\operatorname{Id}$ and so is easily invertible.
      If $P$ is too close to $A^{-1}$, computing the preconditioning will be as expensive as solving the original linear problem.
      If $P$ is too easy to compute, meaning close to $\operatorname{Id}$, tor preconditioning will be useless.
      Choosing a preconditioner means making a compromise between those two extrema.

      \paragraph{}
      Just as Krylov subspace methods only need to compute matrix-vector products, and not to know the matrix, we do not need to know the preconditioning matrix, but only to know how to apply it on any given vector.

      \paragraph{}
       To better understand how preconditioning works, let us take an example, inspired by example 35.2 of \cite{TrefethenBau1997}.
       We take a square matrix of size 200 by 200, which is the sum of a diagonal part and a random perturbations part:
       \begin{equation}
         \begin{aligned}
           A &= A_{diag} + \frac{1}{2\sqrt{N}}A_{rand} \\
           \textrm{where}\quad A_{diag, k} &= 2\sin\left( \frac{2 k \pi}{N - 1} \right) - 1 + i \cos\left( \frac{2 k \pi}{N - 1} \right) \\
           \textrm{and}\quad A_{rand} &\hookrightarrow \mathcal{N}\left(0, 1\right) \ .
         \end{aligned}
       \end{equation}
       The vector $b$ is a vector of ones.
       The spectrum of $A$ is displayed on figure \ref{fig:preconditioning} in blue.
       It is spread out around the origin \PS{(ça se dit en anglais ?)}
       When we apply the Jacobi preconditioning, meaning the preconditioning matrix is the invert of the diagonal part, the spectrum is gathered around 1.
       This helps a lot GMRES, as can be seen in figure \ref{fig:preconditioning}: it struggles to reduce the residual norm on the standard problem but does it easily on the preconditioned problem.
       This is only a dummy example, but it helps visualise the importance of preconditioning.

   		\begin{figure}
   			\centering
   			\includegraphics[width=\textwidth]{figures/preconditioning.png}
   			\caption{Spectrum of $A$ in blue and $A_{pre} = AD^{-1}$ in orange (left). GMRES convergence for those matrices (right).}
   			\label{fig:preconditioning}
   		\end{figure}

      \paragraph{}
      We need to use preconditioners when solving the linear system (\ref{eq:linear}) with GMRES.
      One of the most unsophisticated preconditioners is the Jacobi one.
      It is the one used in the example above: the preconditioning matrix is the invert of the diagonal part of $A$.
      Its advantage resides in its simplicity.
      It is inexpensive and easily \PS{parallélisables}.
      On the other hand, it lacks efficiency, and that is why many preconditioners were developed in the literature.
      A slight variation consists in taking the diagonal blocs instead of the diagonal elements.
      This makes sense for our matrices, as they can be divided into blocs, each block corresponding to the degrees of freedom in a cell.
      The resulting preconditioner, called Block Jacobi, already exists in our solver and is often used as the standard preconditioner.

      \paragraph{}
      Others preconditioners exist in our solver.
      One consists in using the invert of the cell volumes as a diagonal matrix: the preconditioning matrix is a diagonal matrix, where each diagonal element is the inverse of the corresponding mesh cell volume.
      This was originally conceived to preserve the conservative properties through the GMRES solve \PS{clair ?}.
      As it is even simpler than the Jacobi preconditioner, it shows poor performance.
      Some work was done towards Polynomial preconditioning: the preconditioning matrix $P$ is equal to the partial Neumann series of $\operatorname{Id} - A$, as it approximates $A^{-1}$ \cite{DuboisGreenbaumRodrigue1979}.
      Its major drawback is its computational cost, as it requires multiple matrix-vector products.
      For this reason, we decided not to look into it.

      \paragraph{}
      We were at first interested in some other preconditioners found in the literature, such as the Incomplete LU preconditioner.
      When taking the LU factorisation of the sparse matrix $A$, the triangular matrices $L$ and $U$ lose the sparsity pattern of $A$.
      This is troublesome as it is not possible to store dense matrices of the size of $A$.
      The Incomplete LU preconditioner, characterised by its index $k$, computes only the coefficients of the triangular factors that belong to the sparsity pattern of $A^{k+1}$.
      This way, the factorisation is incomplete in the sense that it does not recreate $A$, but it is still sparse and can be used as a preconditioner.
      The Incomplete LU factorisation is often used in computational fluid dynamics \cite{LiuZhangZhongEtAl2015, AhrabiMavriplis2020}, but we decided it was not a good fit for our solver, due to the difficulty in developing the method in an industrial-size solver, and because it relies directly on the matrix.
      This second point will make sense later in the document.
      \PS{En gros on a pas une bonne matrice donc on veut pas qu'elle soit au coeur du précond, et si on fait du JFNK on a même pas la matrice. Est-ce que je peux laisser ça comme ça ou est-ce que c'est pas terrible ?}

      \paragraph{}
      As we do not need explicitly the preconditioning matrix $P$, and its effect should be near $A^{-1}$, one could take as a preconditioning procedure another Krylov subspace method:
      To apply the preconditioning to any vector $v$ would mean solving $Ax = v$ with a Krylov subspace method.
      However, applying a Krylov subspace method is not a linear operator.
      To handle this case, we need to modify the GMRES algorithm into the Flexible Generalized minimal residual method, or FGMRES \cite{Saad1993, SimonciniSzyld2002}.
      We saw several advantages to this method.
      We can precondition GMRES with an inner GMRES, and as GMRES was already written in our solver this idea was simple to do.
      This preconditioning does not need more information on the matrix $A$ than the outer GMRES does: in particular it does not require knowing the matrix coefficients.
      This is interesting to us for some reasons that are discussed later.
      Finally, this method interest many scientists \cite{CoulaudGiraudRametEtAl2013, Vasseur2016} and has shown promising results in numerical simulation \cite{Pinel2010}.
      For all those reasons, we decided to add this method to our solver.

      \paragraph{}
      The GMRES method is often used with restarting: instead of keeping on iterating the method, one could stop it and start again.
      This is the Restarted GMRES method.
      The cost of one iteration increases as the method iterates.
      Restarting allows for some control of the Krylov subspace dimension, and therefore on computational cost and memory usage.
      The issue with restarting is that it can be harmful to the convergence.
      When the Krylov subspace generated on the restart is too close to the previously generated Krylov subspace, the residual norm may stagnate \cite{Simoncini1999}.
      A solution to this problem would be to reuse information between restarts.
      This is done with the Loose GMRES method \cite{BakerJessupManteuffel2005} or with Augmentation or Deflation techniques  \cite{ChapmanSaad1997, Morgan2002, RamosKehlNabben2020}.
      Augmentation and Deflation give in fact equivalent results \cite{CoulaudGiraudRametEtAl2013}.
      Their idea is to recycle the spectral information acquired at the end of the cycle onto the next cycle.
      Recycling spectral information could even be done throughout multiple linear solve, during a nonlinear solve for example \cite{Gaul2014}.
      Those techniques look promising but we did not have time to try them in our solver, as we focused on other points.
      \PS{Je peux dire ça ?}


    \subsection{Evaluating the Jacobian matrices}

      \paragraph{}
      If we can solve precisely linear problems, but the problem is not the one the nonlinear solver wants, it may hurt the nonlinear solver convergence.
      We then need to be able to get the right linear problem from the nonlinear one.
      At this point, we know how to evaluate the function $f$ from equation (\ref{eq:nonlinear}), that is often expressed as a linear combination of previous states and $\operatorname{G}$ evaluations, where $\operatorname{G}$ is the function introduced in equation (\ref{eq:init_value_ode}), the function from our starting ordinary differential equation.
      On the other hand, computing its derivative with regard to $x$ is a different story.
      As this derivative is the matrix we use when solving the linear problem, it is crucial to have a good representation of it.
      This derivative $f'\left(x\right)$ uses the Jacobian matrix of the function $\operatorname{G}$.
      As the function $\operatorname{G}$ comes from the spatial discretisation method applied to the original partial differential equations (\ref{eq:pde}), it is complex, not in the sense of imaginary numbers but of complicated.
      The underlying algorithm is hard to fully understand and write, and the numerical evaluation is usually expensive.
      Furthermore, our software keeps evolving, and models are constantly added and modified.
      It would require constant work to maintain the Jacobian matrix computation.
      Finding an algorithm that computes the exact Jacobian matrix analytically would amount to too much work in our industrial software.
      We must use other alternatives to get the Jacobian matrix we need for Newton's method.

      \paragraph{}
      An idea would be to use \emph{Automatic Differentiation} \cite{Griewank2000}.
      This means to give the source code of the $\operatorname{G}$ function to some software \cite{HascoeetPascual2012}, which gives in return a way to compute its Jacobian matrix.
      One advantage of this method is that the cost of derivating the function is done only once, at software compilation.
      After that, computing the Jacobian matrix amount to calling a function.
      Another advantage is that the given Jacobian is supposedly exact, contrary to some alternatives we will discuss later.
      For those reasons, Automatic Differentiation is today being used in actual computational fluid dynamics software \cite{BilanceriBeuxElmahiEtAl2011, KenwayMaderHeEtAl2019}.
      In our software, using Automatic Differentiation did seem too hard and therefore we looked at other methods instead.
      \PS{Plus justifier ? Reformuler ?}

      \paragraph{}
      A possible idea is to use an approximation of the Jacobian matrix that is inexpensive to compute.
      We said that the function $\operatorname{G}$ comes from the second order or higher Finite Volume method used as the spatial discretisation method.
      We can take the function $\operatorname{G}_1$ given by the first-order corresponding method, and use its Jacobian matrix instead.
      Indeed, computing $\operatorname{G}_1$ is inexpensive in comparison to $\operatorname{G}$, and the same goes for the corresponding Jacobian matrices.
      Going further, we could also approximate the Jacobian matrix of $\operatorname{G}_1$: when $\operatorname{G}$ uses some complex models such as turbulence models, it is often decided to not include those models' contributions to the Jacobian matrix, for complexity and stability reasons \PS{ref ici}.
      This is what is done originally in our in-house solver: using a cheap low-order approximation of the Jacobian matrix.

      \paragraph{}
      When we decided to use Krylov subspace methods as our linear solver, we highlighted the fact that they only need to know how to compute matrix-vector products.
      Using this, and the fact that the matrix is a Jacobian matrix, we could approximate the matrix-vector product of $f'\left(x\right)$ with a vector $v$ by:
      \begin{equation}\label{eq:matrix_free}
        f'\left(x\right) v \approx \frac{f\left(x + \varepsilon v\right) - f\left(x\right)}{\varepsilon}
      \end{equation}
      with a scalar parameter $\varepsilon$ discussed below.
      This is the finite difference approximation of the Jacobian matrix-vector product.
      A quick analysis shows that the error on this approximation is $o\left(\varepsilon\right)$.
      Usually, the value of $f\left(x\right)$ is previously computed, so one evaluation of $f$ is required for one matrix-vector product.
      We could also use the centered finite difference approximation:
      \begin{equation}
        f'\left(x\right) v \approx \frac{f\left(x + \varepsilon v\right) - f\left(x - \varepsilon v\right)}{2\varepsilon}
      \end{equation}
      that gives an smaller error in \PS{article} $o\left(\varepsilon^2\right)$, but it requires two $f$ evaluations, so it is twice as expensive as the first order approximation.
      Therefore, we will use the first-order approximation.

      \paragraph{}
      Using this approximation, it is possible to recover the full Jacobian matrix.
      If we apply the approximation (\ref{eq:matrix_free}) for each vector of the \PS{base canonique} as the vector $v$, we can gather each column of the Jacobian matrix.
      This gives what is often called the finite difference Jacobian matrix.
      Instead of computing a matrix-vector product for each direction, a technique consists of computing independent directions beforehand to get a colouring of the matrix: two directions are of the same colour if they are independent through the matrix.
      Then, a smaller number of matrix-vector product approximations are required: as many as there are colours.
      This is often called the finite difference Jacobian matrix with colouring \cite{GebremedhinMannePothen2005}.
      Those techniques are still not well fitted for our solver.
      As we work with large dimensions, computing the Jacobian matrix takes time.
      Furthermore, we explained that we do not need to know explicitly the matrix.
      That is why we will use the approximation (\ref{eq:matrix_free}) each time we need a matrix-vector product instead of computing the Jacobian matrix first.

      \paragraph{}
      We still have not talked about the parameter $\varepsilon$ introduced in equation (\ref{eq:matrix_free}).
      As it represents the size of the step made to approximate a derivative, it should be small.
      But taking it too small leads to roundoff errors so it must be chosen carefully \cite{KnollKeyes2004}.
      We decided to look at some strategies in the choice of $\varepsilon$, and this work will be presented in a later part.

      \paragraph{}
      Now the choices we made for our nonlinear and linear solve strategies are starting to make sense.
      Working with different Jacobian matrices is not an issue, as they are not computed.
      The Krylov subspace method GMRES does not need the Jacobian matrix, only to compute matrix-vector products, contrary to other iterative methods such as the Gauss--Seidel method.
      The flexible preconditioning also does not need the Jacobian matrix for the same reason, contrary to other types of preconditioners such as ILU preconditioners.

      \paragraph{}
      Overall, using Newton's method to solve the nonlinear problem and a Krylov subspace method to solve the linear problem while using a matrix-free method is known as the Jacobian-Free Newton--Krylov method, or JFNK.
      This methods and its variants are as of today still discussed \cite{AnWenFeng2011, Turpault2003} and used on actual computational fluid dynamics solvers \cite{LiuZhangZhongEtAl2015, FrancoCamierAndrejEtAl2020}.
      We are interested in the JFNK method as we hope it will give a better representation of the Jacobian matrix than the low-order one already in use in our solver.
      We hope that a better Jacobian matrix will lead to a better convergence, in particular when some models are ignored in the classic Jacobian matrix, such as turbulence models.
      Furthermore, CEDRE works with multiple distinct solvers.
      When a computation uses multiple solvers, two for example, the function $f$ from equation (\ref{eq:nonlinear}) is in fact $\left(f_1, f_2\right)$ and $x$ is $\left(x_1, x_2\right)$.
      The Jacobian matrix is then:
      \begingroup
      \renewcommand*{\arraystretch}{1.5}
      \begin{equation}
        f'\left(x\right) = \begin{pmatrix} \frac{\partial f_1}{\partial x_1} & \frac{\partial f_1}{\partial x_2} \\ \frac{\partial f_2}{\partial x_1} & \frac{\partial f_2}{\partial x_2} \end{pmatrix}\ .
      \end{equation}
      \endgroup
      The cross-term Jacobian matrix, $\frac{\partial f_i}{\partial x_j}$ with $i \ne j$, are usually hard to compute, both analytically and computationally \PS{c'est vrai ?}, and are the main obstacle to a fully implicit solver.
      Instead, in CEDRE, each solver iterates independently to the others, with some exchanges between iterations.
      Using the matrix-free approximation and the JFNK method, those cross-solver terms are taken into account.
      This choice is then perfectly aligned with the direction our solver is aiming: towards a better implicit cross-solver mechanism.


  % \chapter{Implémentation d'une méthode JFNK dans CEDRE}

  % \chapter{Analyse de la méthode JFNK dans CEDRE}
  %   \section{Comparaison entre la matrice jacobienne explicite et la formulation sans matrice}
  %     \subsection{Sphère hypersonique}
  %     \subsection{Profil RAE dans un écoulement transsonique turbulent}
  %   \section{Utilisation de la formulation sans matrice sur un nouveau modèle de fluide}
  %     \subsection{Modèle Multi-températures}
  %     \subsection{Sphère hypersonique}
  %       \subsubsection{Modèle réactif complet}
  %       \subsubsection{Modèle réactif simplifié}


% \part{Intégration temporelle à grand pas de temps}

  % \chapter{Analyse d’une nouvelle catégorie d’intégrateurs temporels}

  % \chapter{Analyse d’une nouvelle catégorie d’intégrateurs temporels}
  %   \section{Présentation de \emph{Jaguar}}
  %   \section{Analyse}
  %     \subsection{Analyse de l'ordre : convection d'un tourbillon isentropique}
  %     \subsection{Analyse de la robustesse : tourbillon de Taylor--Green}
  %     \subsection{Analyse de la rapidité : LS89}



\pagebreak
\bibliography{bibliography.bib}
\bibliographystyle{ieeetr}
\addcontentsline{toc}{chapter}{Bibliographie}

\end{document}
