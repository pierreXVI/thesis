\chapter*{Introduction}
\addcontentsline{toc}{chapter}{Introduction}


  \section*{General context}

    \paragraph{}
    This work is rooted in the field of numerical simulation of fluid dynamics, applied to \PS{energy/energetics}.
    This field gathers many industrial actors (the French Directorate General of Armaments, ArianeGroup, Safran, Airbus, etc) as well as academics (ONERA, CERFACS, DLR, VKI, universities, etc).
    It is interested in how to simulate a fluid flow with a computer, trying to represent the physical reality faithfully.
    The various actors in this field need to be able to access certain physical quantities associated with specific phenomena and operating regimes.
    These regimes are often not feasible on our scale, due to material or financial limitations.
    Examples include the study of icing on the wing of an aircraft, which is experimentally feasible but represents an imposing budget for the aircraft manufacturer, or the study of heat transfer in an atmospheric reentry capsule, which is much more difficult to achieve experimentally.
    To overcome these limitations, numerical simulation is the best option, as it allows such a case study to be modelled by the execution of a computer program, and to obtain a large set of data that will be analysed afterwards to answer the desired questions.

    \paragraph{}
    The analysis of physics usually produces a set of equations, often partial differential equations, representing the real system that one wishes to study.
    Algorithms are then required to determine the fluid flow from these equations, in the working domain, as a function of time.
    Thus, to obtain the desired quantities, the physical system is integrated in time using mathematical algorithms to obtain its evolution in time.
    Often, the expected result is not the complete temporal evolution of the system, but only its equilibrium state.
    This is called a steady numerical simulation or steady computation.
    Steady computations are opposed to unsteady computations that aim to accurately describe the system temporal evolution.

    \paragraph{}
    For all computational fluid dynamics users, the most crucial issue is to achieve a satisfying compromise between computational cost and accuracy of the results.
    Indeed, a quick and inexpensive computation tends to be not very faithful to physics, while an accurate computation tends to use more computational ressources and time.
    For steady computations, speed corresponds to getting the final steady state at a low time cost, for both computational time and the time it actually took to a user.
    It results in a compromise between methods that are expensive in terms of computational ressources and take a long time but give accurate results that are close to the physical reality, and faster methods that saves ressources but give lesser quality results.
    A software developer working on a numerical simulation tool needs to choose methods and algorithms to obtain a compromise that he considers satisfactory.
    The final interest for a player in computational fluid dynamics is therefore to have a result that is sufficiently precise and inexpensive enough to obtain.
    An accurate result is needed to answer the questions that required the simulation.
    The search for a result that is inexpensive to obtain is motivated by questions of savings in computational cost.
    This cost is applied to the user, while he wait for the simulation to end, and to its company through the cost in computer ressources, electricity, investments in more efficient machines, etc.

    \paragraph{}
    Depending on the problem to be solved, there are more or less suitable algorithms and methods.
    Performant methods were originally developed to answer needs from the aerodynamics community.
    In the case of energetic and multi-physics problems, the methods issued from the more traditional aerodynamics are limited by the coupling  between the different physics which have their distinct characteristic times.
    Thus, the algorithms used for the numerical simulation of classical fluid dynamics, i.e. concentrated on the resolution of the Navier--Stokes equations, are not necessarily the most adapted to a simulation in the energetic domain.
    The involvement of several distinct physical phenomena imposes constraints on the choice and use of algorithms.
    Consequently, it would be advisable to adapt or replace the algorithms involved in the time integration for multiphysics problems.

  \section*{Études}

    \paragraph{}
    Si la simulation numérique s'est grandement développée dans le domaine l'aéronautique, elle ne s'est pas tant adaptée au domaine de la multi-physique, et de nombreux codes industriels se contentent de réutiliser les mêmes algorithmes.
    C'est l'exemple du code CEDRE, développé à l'ONERA par le Département Multi-Physique pour l'Énergétique.
    Ce code constitue une plateforme regroupant plusieurs solveurs pour intégrer plusieurs physiques : chaque solveur est dédié à son modèle physique.
    On compte alors un solveur pour la résolution des écoulements compressibles, multi-fluides, réactifs et turbulents, deux solveurs pour le calcul de phase dispersée (gouttes, cristaux, particules) en approche eulérienne et lagrangienne respectivement, un solveur dédié au calcul des films liquides, un solveur dédié au rayonnement, ...
    Des travaux on été fais pour mettre en place une intégration temporelle adaptée aux problèmes résolus par CEDRE.
    C'est par exemple le cas de \cite{Selva1998}.
    Un travail sur l'intégration temporelle a mené au développement de méthodes d'intégrations implicites pour l'intégration des problèmes stationnaires, et au développement d'une méthode GMRES pour la résolution des systèmes linéaires.
    Ainsi, CEDRE constitue en fait un solveur global adapté aux problèmes multi-physiques.
    L'utilisateur peut choisir parmi un panel de méthodes d'intégrations pour obtenir une méthode adaptée à son problème.
    Cependant, la couplage faible entre solveurs entache la convergence vers l'état stationnaire et le choix dans les méthodes d'intégration est limité, en comparaison à ce qu'on peut trouver dans la littérature.
    C'est du moins l'avis des acteurs du code CEDRE, c'est à dire ses développeurs et ses utilisateurs, qui aimeraient des méthodes plus robustes pour pouvoir utiliser CEDRE sur des problèmes plus raides, et convergeant plus vite pour économiser en coût de calcul.

    \paragraph{}\PS{TODO}
    Du coté de la recherche, cependant, de nombreux efforts ont été réalisés dans le sens de la multi-physique, mais ne sont pas encore sorti du cadre académique.
    C'est par exemple le cas de \cite{WongKwokHorneEtAl2019}, qui se sont intéressés à l'intégration temporelle d'équations couplées.
    Ils ont conçu une méthode d'intégration adaptée aux équations couplées, qui est une évolution d'une méthode du point fixe avec en plus une étape d'une méthode de Newton.
    Ils ont ensuite comparés cette méthode à la méthode du point fixe standard, plus généralement utilisée pour une résolution couplée.
    En mettant en place leur méthode sur deux problèmes de couplage simple, ils ont enfin montré l’intérêt de leur méthode par rapport à la méthode de base.
    Si cette méthode se prête bien à leur calculs, elle n'est cependant mise en place que pour des problèmes simples, moins complexes que les problèmes multi-physiques que CEDRE souhaite résoudre.
    De plus, les tests réalisés sont sur des problèmes à l'échelle académique, et non industrielle.

    \paragraph{}
    Parallèlement, des outils utilisés dans le cadre de la simulation aérodynamique pourraient s'avérer intéressants pour des problèmes multi-physiques.
    La méthode JFNK est déjà bien utilisée dans la simulation numérique des équations de Navier-Stokes.
    Dans \cite{ParkNourgalievMartineauEtAl2009}, un mécanisme d'intégration temporelle est mis en place autour d'une formulation JFNK.
    Un préconditionnement basé sur la physique est développé pour résoudre plus précisément le problème linéaire.
    Cette intégration temporelle est ensuite testée sur un problème de cavitée carée thermiquement entrainée.
    Cependant, cette méthode ne s'intéresse seulement aux équations de Navier-Stokes et n'est pas directement appliquable aux problèmes plus généraux de l'énergétique.
    De plus, elle n'est testée que sur un problème académique 2D.

    \PS{TODO: quand parler de la distinction CEDRE / CHARME, placer \cite{ReflochCourbetMurroneEtAl2011}}

  \section*{Bilan général}

    \paragraph{}
    On voit donc que des méthodes numériques sont déjà disponibles pour résoudre des problèmes de simulation numérique.
    En particulier, il existe déjà des solveurs capables de résoudre les problèmes stationnaire multi-physiques d'échelle industrielle.
    D'un autre coté, d'autres méthodes développées dans un cadre académique ont montré leur intérêt sur des problèmes multi-physiques simples.

    Enfin, des méthodes améliorant la résolution des problèmes d'aérodynamique pourraient s'avérer intéressantes pour des problèmes multi-physiques.

    Enfin, certaines méthodes récentes ont permis ...

    Les solveurs déjà existants utilisent des méthodes anciennes, ou peu adaptées aux problèmes multi-physiques.
    Les méthodes plus performantes ne sont utilisées que sur des problèmes simplifiés, ou dans des calculs d'échelle académique.

    Il semblerait maintenant intéressant d'adapter ces nouvelles méthodes pour la résolution des problèmes multi-physiques dans un code industriel.

    \vspace{1cm}\hrule\vspace{1cm}

  \section*{C'est ce qui justifie cette étude ...}

    \paragraph{}
    ..., elle consiste à améliorer la convergence, la rapidité et la robustesse de l'intégration temporelle de la plateforme CEDRE sur les problèmes stationnaires multi-physiques en ajoutant des méthodes numériques non utilisées dans la simulation numérique industrielle.


  \section*{Démarche}

    \paragraph{}
    L'objectif du chapitre 1 a été de développer une formulation JFNK pour AMÉLIORER QUOI DE l'intégration temporelle dans le code CEDRE.
    JUSTIFICATION
    Pour cela, l'idée a été d'adapter la méthode d'intégration en s'inspirant de la bibliographie en fonction de certains critères.
    L'idée suivante a été de développer une méthode de Newton afin de résoudre le problème non linéaire sachant que celui ci est produit par les méthodes d'intégration implicites utilisées lors de la résolution des problèmes stationnaires.
    L'idée suivante a été de développer une formulation sans matrice du système afin de mieux former les systèmes linéaires et permettre un couplage fort des solveurs de CEDRE.
    L'idée suivante a été d'adapter la résolution du système linéaire en utilisant des méthodes plus modernes que la méthode actuelle, en s'inspirant de la bibliographie, afin de résoudre plus précisément le système linéaire.
    On a alors obtenu une méthodologie d'intégration temporelle utilisable pour résoudre des problèmes multi-physique stationaires.
    A ce stade, cette méthode fonctionne au sens où elle permet d'obtenir un résultat à un problème de simulation numérique, mais elle n'est pas encore caractérisée.

    \paragraph{}
    L'idée du chapitre 2 a été d'évaluer la robustesse et la convergence de la formulation JFNK sur des problèmes stationnaire en multi-physiques au sein du code CEDRE.
    Pour cela, l'idée a été de sélectionner des cas tests de complexité croissante afin de représenter un ensemble de problèmes multi-physiques.
    L'idée suivante a été de mettre en place une méthode d'évaluation des performances de l'intégration temporelle afin de caractériser sa robustesse et sa vitesse de convergence.
    L'idée suivante a été de montrer que sur ces cas la formulation améliore la robustesse et la convergence afin de valider la pertinance du choix de la méthode.
    On estime qu'on a alors développé une méthode d'intégration temporelle "efficace", c'est à dire rendant l'intégration temporelle de CEDRE plus robuste et convergeant plus rapidement sur un ensemble de problèmes multi-physiques.
    On pourrait aller plus loin mais on va plutôt regarder l'intérêt de la formulation JFNK sur des problèmes autres pouvant en tirer profit : les problèmes instationnaires à grand pas de temps.

    \paragraph{}
    L'idée du chapitre 3 est d'analyser la formulation JFNK sur les problèmes instationnaires à grand pas de temps.
    Pour cela, l'idée a été d'identifier une classe de problèmes sortant du cadre initial de la thèse mais pouvant bénéficier de la formulation JFNK, afin d'y démontrer l'intérêt de la formulation.
    L'idée suivante a été de concevoir un cas d'étude afin de de mettre en valeur l'intérêt de la formulation JFNK.
    L'idée suivante a été de montrer l’intérêt la formulation sur ce cas.
