\chapter{Analysis of existing methods}

  \section{Problem setup}

    \paragraph{}
    In this part we are going to set the mathematical framework for this study.
    We will start from a partial differential equation arising from the physical model, in the form of
    \begin{equation}\label{eq:pde}
      \frac{\partial \xi}{\partial t} + \operatorname{F}\left(\xi\right) = 0
    \end{equation}
    where the function $\operatorname{F}$ uses some space derivatives of the state variable $\xi$.
    This equation then describe the temporal evolution of the state variables $\xi$.

    \paragraph{}
    A particular class of such partial differential equations are conservative equations.
    They correspond to the case where the function $\operatorname{F}$ can be written as a divergence term.
    Finally, with a source term $\operatorname{S}$, those equations look like:
    \begin{equation}\label{eq:pde_conservative}
      \frac{\partial \xi}{\partial t} + \nabla \cdot \operatorname{f}\left(\xi\right) = \operatorname{S}\ .
    \end{equation}
    One might notice that equation (\ref{eq:pde_conservative}) is indeed a particularisation of equation (\ref{eq:pde}), with $\operatorname{F}\left(\xi\right) = \nabla\cdot \operatorname{f}\left(\xi\right) - \operatorname{S}$.
    Those conservative equations are the one we will focus on is this study, as they describe the physical systems we are interested in.

    \paragraph{}
    When solving numerically equations like (\ref{eq:pde}), one must first take a spatial domain on interest.
    Let us call this domain $\mathcal{D}$.
    We need to be able to represent different quantities, such as the state variable $\xi$ numerically over the domain $\mathcal{D}$.
    As we are interested in solving equations numerically, we need to be able to represent different quantities, such as the state variable $\xi$, numerically over the domain $\mathcal{D}$ and store it in the memory of a computer.
    Therefore we need to discretise the continuous spatial domain into a finite number of cells, or elements.
    This is usually done with a mesh of the domain $\mathcal{D}$.
    First we divide the domain $\mathcal{D}$ in a set of cells, called a mesh.
    Those cells are small volumes in 3D, faces in 2D or segments in 1D, disjoints, such as their union recovers the original domain.
    Interest quantities, such as the fluid velocity, density, \dots, are then stored at each nodes, averaged in the center of each cells or sometimes in a more complex fashion depending on the method.
    Quantities are then stored at the nodes, the center of the cells, or sometime in a more complex fashion depending on the method used.
    They are no longer mathematically represented by a function of the continuous physical domain $\xi : \operatorname{D} \rightarrow \mathbb{R}$ but by a finite sized vector $\Xi$ gathering all the information across the discretised domain.
    For some simple discretisation methods, this vector consists of the quantity evaluated at the mesh nodes or averaged at the center of the cells.
    For more complex methods, this vector consists on information used to construct the solution over the domain : polynomial coefficients, spectral decomposition coefficients, etc.
    Anyway, we no longer work in a continuous domain $\operatorname{D}$ but on a discretised one.

    \paragraph{}
    The partial differential equation (\ref{eq:pde}) transforms then into an ordinary differential equation:
    \begin{equation}\label{eq:ode}
      \frac{\partial \Xi}{\partial t} + \operatorname{G}\left(\Xi\right) = 0 \ .
    \end{equation}
    The difference here is that the function $\operatorname{G}$ is a function of a discrete vector whereas $\operatorname{F}$ was a function of a continuous function, and therefore $\operatorname{G}$ does not uses any spatial derivatives.
    Thanks to the spatial discretisation method, the only derivative remaining is with regard to time.
    The rest is then up to the temporal integration method, which is the main topic of this thesis.
    We will work from equation (\ref{eq:ode}) no matter where the function $\operatorname{G}$ comes from, but sometimes understanding the origin of this function can help so we will now introduce the spatial discretisation method used in our solver.



    \paragraph{}
    Bien que cette thèse porte sur l'intégration temporelle, il est bon de s'intéresser également à l'intégration spatiale.
    L'\emph{intégration spatiale} désigne l'ensemble de méthodes utilisées pour calculer une dérivée spatiale d'une grandeur physique.
    Puisque la résolution des équations est numérique, il faut pouvoir représenter les données dans la mémoire d'un ordinateur.
    Par une étape de discrétisation spatiale, on divise le domaine d'étude en un ensemble de cellules qui forment un maillage.
    Ces cellules sont des petits volumes \footnote{ou des faces en 2D, ou des segments en 1D} disjoints, telles que leur réunion forme le domaine d'étude.
    Les grandeurs physiques étudiées comme la vitesse du fluide, sa température, sa densité, \dots, sont alors représentées dans chaque cellule par leur valeur moyenne, leur valeur en chaque noeud de la cellule, ou de manière plus complexe en fonction du choix de discrétisation pris par l'utilisateur.
    L'état physique dans l'ensemble du domaine est représenté non plus par une fonction continue de l'espace mais par un vecteur discret d'états
    On utilise une telle discrétisation spatiale pour pouvoir représenter l'état sur un espace discret plus petit que l'espace continu, mais également car elle fourni des moyens de calculer des dérivées spatiales, nécessaires à la résolution des équations aux dérivées partielles.


    \subsection{Finite Volumes method}

      \paragraph{}
      La méthode d'intégration spatiale utilisée dans CHARME est la méthode des Volumes Finis \cite{EymardGallouetHerbin2000}.
      Cette méthode a été développée principalement pour les équations conservatives, c'est à dire qui conservent les intégrales des différentes quantités sur l'ensemble du maillage.
      En effet, de telles équations font généralement apparaitre la divergence d'un flux, et la méthode des Volumes Finis a pour propriété de calculer des flux conservatifs.
      Ainsi, la propriété de conservativité de l'équation est bien respectée par la méthode d'intégration spatiale.

      \paragraph{}
      La méthode des Volumes Finis consiste à intégrer l'équation à résoudre dans chaque cellule du maillage.
      Ainsi, l'état dans chaque cellule est représenté par sa valeur moyenne dans cette cellule, et le théorème de Green--Ostrogradski transforme la divergence du flux en un bilan surfacique.







    \subsection{Gradient reconstruction methods}
      \subsubsection{The k-exact method}
      \subsubsection{The Multislope method}


  \section{Time integration methods}
    \subsection{Analyse des méthodes}
      \subsubsection{Consistance et ordre}
      \subsubsection{Stabilité}
    \subsection{Méthodes implicites}
      \subsubsection{Méthode d'Euler implicite}
      \subsubsection{Méthodologie des méthodes implicites}
