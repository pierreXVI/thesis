\chapter*{Conclusion and perspectives}

  \paragraph{}
  This thesis was interested in solving efficiently steady problems for multiphysic applications.
  It means improving the convergence, stability and speed of already existing methods, particularly for the software system CEDRE.
  A preliminary analysis of existing methods whithin CEDRE and the literature focussed our effort on improving the nonlinear and linear resolutions required by implicit time integration methods.
  In particular, the Jacobian-Free Newton--Krylov method caught our interest.
  This is first because it reuses many algorithms that were already available in CEDRE.
  The other reason is that it is a better alternative to what was currently done, which is using a poor approximation of the Jacobian matrix.
  Indeed, using the JFNK method amounts to using a full Jacobian matrix that accounts for all numerical models.
  Our hope was that using a more accurate matrix for the linear problem would improve the quality of the nonlinear solution, which would in turn improve the overall quality of the implicit time integration method.
  With a carefull implementation, it was added to the software system to ensure compatibility with other solvers and existing algorithms.

  \paragraph{}
  To check the quality of the Jacobian-Free Newton--Krylov method, it was tested on several applications.
  Those applications were choosen to represent typical CEDRE computations, to evaluate if the method is interesting for the solver.
  The first application was the simulation of the turbulent flow around a wing profile in two-dimensions.
  When compared with the traditional method, the new one that is matrix-free gives similar results when looking at flow data.
  The improvement is solely on the convergence, as the JFNK method converges.
  The second application is the same wing profile, but this time using a much finer mesh to finely represent the boundary layer.
  This time, the traditional method lack the ability to converge satisfyingly when the JFNK method, however, does.
  It shows that the JFNK method improves the convergence of the solver on such applications.

  \paragraph{}
  As CEDRE aims to solve problems in the field of multiphysics, it is not enough to test the new method on aerodynamics test cases.
  This is why the next choosen application was the simulation of the hypersonic reactive flow around a solid sphere.
  It is a simplified typical reentry application.
  Similar results are found when looking at the non-reactive equivalent of this test case: the JFNK method convergence is better.
  However, when using fine mesh with the reactive model, the conclusion is no longer in favor of the matrix-free method as the older method converges better.
  Still, the difference in the convergence between the two method is small compared to the same difference on the non-reactive case.
  A possible conclusion is that on the best case scenario, when the traditional method converges well, the JFNK does not improve the quality of the solution.
  When the traditional method struggle to find a steady solution, however, the JFNK method becomes interesting.

  \paragraph{}
  Finally, the Jacobian-Free Newton--Krylov method was tested on a newly implemented fluid model.
  Indeed, as CEDRE is under current developpement, new fluid models are sometimes added to represent flow features with higher fidelity.
  For instance, the new model used here accounts more precisely for thermodynamic disequilibriums.
  Such newly implemented model often do not give access to their Jacobian matrices, as it would require much more work from developpers and is not a priority.
  It means that users must restrict themselves to using explicit time integration methods.
  As the JFNK method does not require Jacobian matrices, it is a good candidate to be used as a time integration method with such models.
  It was then compare with the explicit Midpoint method.
  In the end, the JFNK method converges better than the explicit method, and is also quicker in terms of CPU time.
  The result is even more in favor of the implicit method when using a simplified version of the reactive model that disregards ionisation.

  \paragraph{}
  In the end,
