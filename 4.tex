\chapter{Analysis of a new type of time integration methods}

  \paragraph{}
  In this thesis, we were interested in finding solutions to steady problems.
  As we explained in the previous part, we use implicit time integration methods to get efficiently solutions of such problems.
  This is a pretty standard choice: most computational fluid dynamics solvers use implicit time integration methods to solve steady problems.
  The reason is because they can use larger time steps than their explicit counterparts.
  Because of this advantage, they are also often used when solving unsteady problems but with large time steps.
  Indeed, solving an unsteady problem with large time steps is similar to solving a steady problem.
  Most of what is done towards the steady problem solve can therefore be reused for unsteady problem solves with large time steps.

  \paragraph{}
  Even if implicit time integration methods are quite standard when solving problems with large time steps, there are other less conventional methods we could choose from.
  We could step out from the explicit implicit dichotomy, and decide to use IMplicit-EXplicit, or IMEX, methods.
  They split the function from the ordinary differential equation (\ref{eq:ode}) in two parts: the stiff part than is integrated by an implicit method, because of its stiffness, and the other part that can just be integrated with an explicit method.
  The Additive Semi-Implicit Runce--Kutta methods, or ASIRK methods, are such methods \cite{Zhong1996}.
  They are already in use in computational fluid dynamics problems, such as fluid-structure interaction problems \cite{HuangPerssonZahr2019}.
  Previous work already implemented ASIRK methods in our solver CEDRE for specific multiphysic applications.
  Some methods are even less common and correspond to a total paradigm shift: the parallel time integration methods \cite{Nievergelt1964, LionsMadayTurinici2001}.
  Just as we classically split the computational domain over processes and compute the spatial discretisation method in parallel, parallel time integration methods decompose the time integration interval into subintervals.
  They then solve the ordinary differential equation on each interval concurrently then ensure the continuity between the subintervals.
  They then iterate with Newton's method to find the solution over the whole time integration interval.
  As a result, they can approximate accurately the solution at later time without knowing accurately the solution at previous time.
  Despite being nontraditional, they were successfully used to solve fluid structure interaction problems or Navier--Stokes problems \cite{GanderVandewalle2007}.
  They were even more recently used to solve simple turbulent flow problems \cite{Lunet2018}.
  They can even be used in a more convoluted way, with for instance exponential methods \cite{GanderGuettel2013}.
  Just as parallel time integration is inspired from parallel spatial discretisation, some time integration methods are inspired from spectral discretisation methods.
  Time spectral methods, that were originally used for fluid dynamic time periodic problems \cite{GopinathJameson2005, GopinathJameson2006}, are now used on non-periodic problems \cite{EkiciDjeddiLiEtAl2020}.
  But as both time parallel integration methods and time spectral methods are considered modern methods and highly unconventional, they do not seem appropriate for an industrial solver such as ours.


  \section{Exponential integration methods}

    \paragraph{}
    We decided to look at an other class of time integration methods: exponential methods.
    Despite being known for a long time \cite{Pope1963}, they were not widely used in computational fluid dynamics, because of some difficulties that we will explain later, but still interested some \cite{EdwardsTuckermanFriesnerEtAl1994}.
    They started to come back in the literature \cite{HochbruckOstermann2005} and are now being use on applications similar to ours \cite{NieZhangZhao2006, BhattKhaliqWade2018}.

    \paragraph{}
    We start from an ordinary differential equation
    \begin{equation}\label{eq:ode_2}
      \frac{\mathrm{d} y}{\mathrm{d} t} = f\left(y\right) \ .
    \end{equation}
    This is the same as the ordinary differential equation (\ref{eq:ode}) from the previous part but with different notations.
    The main idea of exponential integration methods is to start from the ordinary differential equation (\ref{eq:ode_2}) and split the function $f$ in a linear part and a nonlinear part:
    \begin{equation}
      \frac{\mathrm{d} y}{\mathrm{d} t} = Ly + N\left(y\right) \ .
    \end{equation}
    This decomposition is sometimes natural for some particular equations, but in the more general case it is always possible: it consists in choosing a linear part $L$ then setting the nonlinear part $N\left(y\right) = f(y) - Ly$.
