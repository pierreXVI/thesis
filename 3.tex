\chapter{Analysis of the Jacobian-Free Newton--Krylov method in CEDRE}
\chaptermark{Analysis of the JFNK method in CEDRE}

\begin{tcolorbox}[title=Résumé du chapitre : Analyse de la méthode JFNK dans CEDRE, colframe=black!50!white]
  \paragraph{}
  Le but de ce chapitre est de vérifier les gains apportés par l'ajout de la méthode JFNK dans le solveur CEDRE.
  Il s'agit de comparer les performances de cette nouvelle méthode avec les méthodes préexistantes en matière de stabilité, robustesse et rapidité.
  Nous adaptons pour cela le point de vue d'un utilisateur, en considérant ce qui l'intéresse lorsqu'il réalise une simulation stationnaire.

  \paragraph{}
  Pour cela, nous comparons tout d'abord la méthode JFNK avec la méthode utilisant explicitement la Jacobienne.
  Cette comparaison se traduit principalement par la comparaison de la convergence des deux méthodes à travers la norme des résidus.
  Ainsi, nous réalisons cette comparaison sur une succession de cas tests, de complexité croissante, sélectionnés car ils représentent différentes facettes du solveur.

  \paragraph{}
  Nous commençons par un calcul purement aérodynamique d'un profil d'aile en deux dimensions dans un écoulement turbulent, qui utilise une modélisation des effets turbulents de la couche limite.
  Puis, nous utilisons le même profil mais cette fois en incidence, avec un maillage bien plus fin qui capture les effets de couche limite turbulente, plutôt que de la modéliser.

  \paragraph{}
  Nous regardons ensuite un cas de rentrée atmosphérique avec une sphère placée dans un écoulement à haute énergie.
  Cela se traduit par un très fort choc, ainsi qu'une zone de déséquilibre thermodynamique où ont lieu d'intenses réactions chimiques.
  Dans un premier temps, nous regardons comme précédemment les deux méthodes pour comparer leur convergence.
  Pour finir, nous utilisons un modèle de fluide récemment ajouté au solveur afin de représenter plus finement le déséquilibre thermodynamique en caractérisant l'écoulement par plusieurs températures distinctes.
  Avec les développements réalisés actuellement sur ce nouveau modèle, il n'est pas possible d'utiliser la méthode implicite préexistante, et les utilisateurs sont forcés d'employer des méthodes explicites.
  Nous montrons donc que la méthode JFNK peut elle être utilisée en l'état, sans nécessiter de développements supplémentaires, et permet d'accélérer les calculs ainsi que d'obtenir une meilleure convergence.
\end{tcolorbox}


  \paragraph{}
  In a previous chapter, we identified some methods from the literature that we wanted to use in our solver CEDRE, and some others from CEDRE that we wanted to improve.
  In another chapter, we discussed the practical implementation of said methods in the solver.
  The goal of this chapter is now to test those methods on several applications to comment on the choices we made.
  We need to define test cases that represent well enough target applications so we can comment on the performances of our choices.


  \section{Comparison between matrix-free formulation and Jacobian matrix approximation}

    \paragraph{}
    From the previous analysis and implementation, the main addition to the solver is the Jacobian-Free Newton--Krylov method, and in particular the matrix-free approach.
    In this part, the new method that uses the matrix-free approach will be compared with the implicit Euler method as the interest is focused on implicit time integration schemes.
    The traditional method linearises the equation that the implicit Euler method produces, approximates the Jacobian matrix using the Jacobian matrix of the corresponding first-order scheme, and then solves the linear system with the Krylov subspace method GMRES.
    In order to understand the impact of a better Jacobian matrix, the only difference is how the Jacobian matrix is handled.
    The new method will then work in a similar way, except the matrix used in the linear solver is not actually computed, but the matrix-vector products are approximated using equation (\ref{eq:matrix_free}).


    \subsection{Turbulent transonic airfoil}

      \subsubsection{Definition of the test case}

        \paragraph{}
        The first application is a typical aerodynamics test case.
        It is a two-dimensional simulation of the flow around an RAE 2822 wing profile.
        The fluid is standard air assumed to be a perfect gas.
        The Mach number is taken equal to 0.75, the chord is equal to $1\si{\meter}$, the angle of attack is $0\si{\degree}$ and we use the atmospheric conditions at 10km.
        This gives a laminar Reynolds number of \num{6.5e6}.

        \paragraph{}
        This first test case is chosen for multiple reasons.
        Firstly, it is a simple case in the field of computational fluid dynamics.
        It is a standard aerodynamics case, with a small mesh in comparison to many other three-dimensional cases.
        This allows testing our methods inexpensively.
        Secondly, this case belongs to the tutorial suite of our solver.
        It means that it is already well mastered by the team.
        Thirdly, even if it is only a standard aerodynamics case it still has some stiff features, such as turbulence modelling and a shock.
        Finally, it is a standard test case for turbulence modelling validation.
        Therefore there are many references in the literature using this case.

        \paragraph{}
        The mesh used for this simulation is an unstructured hybrid mesh made of triangles and quadrangles.
        Parts of it can be seen in figure \ref{fig:rae_mesh}.
        Cell sizes range from $2.5\si{\meter}$ far from the airfoil and $100\si{\micro\meter}$ at the wall.
        At the wall, there is a C-shaped layer of regular cells.
        This helps better capture boundary layer effects near the profile and the wake.
        Also, under those conditions, a shock is expected to develop on the upper part.
        Special treatment such as refinement was applied to the mesh at the expected shock location.

        \begin{figure}
          \centering
          \includegraphics[width=\textwidth]{figures/rae_mesh.png}
          \caption{Mesh for the RAE 2822 test case. Close up on the leading edge and on the expected shock location.}
          \label{fig:rae_mesh}
        \end{figure}

        \paragraph{}
        The model used is the Reynolds Averaged Navier--Stokes equations, or RANS.
        Simply put, every scale of the turbulence is modelled.
        The Spalart--Allmaras turbulence model is chosen to close the RANS equation.
        It is well known for being one of the simplest turbulence models, which is fine to set up a simple first test case.
        Figure \ref{fig:rae_mesh} shows that the mesh near the wall boundary is not particularly fine, and therefore it is not fine enough to compute the boundary layer.
        It is indeed because this computation uses a turbulence wall model that replaces the standard no-slip boundary condition with a more complex relationship between the variables and their derivatives.
        The physics of the boundary layer is introduced in the model, which limits the stiffness of the system.
        Such models are known to be troublesome with some more complex turbulence models such as the $k-\epsilon$ model but are used with others such as the one we use here.
        Turbulence wall modelling may be nonconventional in aerodynamic simulations but it is a commonly used feature of our solver, so it is interesting to see how our new method behaves on such test cases.
        The spatial discretisation method is a second-order Finite Volume method, using the HLLC Riemann solver and Multislope method \cite{LeTouzeMurroneGuillard2015} with a Van--Leer slope limiter.
        Local time-stepping is used to speed up the convergence.


      \subsubsection{Analysis of the results}

        \paragraph{}
        We are now going to compare the results from the two different simulations.
        It is first worth noting that before trying the JFNK method, the computation needs to start using the traditional method.
        As explained before, the matrix-free approximation is not ideal to handle discontinuities such as shocks, and a shock is indeed present in this computation.
        Only some iterations are needed, just to start the development of the boundary layer and approximately place the shock.
        After that, the computation can be continued, with the traditional method on one side and with the new method on the other.

        \begin{figure}
          \centering
          \includegraphics[width=\textwidth]{figures/rae_field.png}
          \caption{Pressure around the RAE 2822 airfoil and sonic $\operatorname{Ma} = 1$ contour.}
          \label{fig:rae_field}
        \end{figure}

        \begin{figure}
          \centering
          \includegraphics{figures/rae_cp.png}
          \caption{Pressure coefficient around the airfoil for the traditional method (blue) and the JFNK method (orange).}
          \label{fig:rae_cp}
        \end{figure}

        \begin{figure}
          \centering
          \includegraphics{figures/rae_coefficients.png}
          \caption{Aerodynamic coefficients (lift and drag) for the profile throughout the computation.}
          \label{fig:rae_coefficients}
        \end{figure}

        \begin{figure}
          \centering
          \includegraphics{figures/rae_residuals.png}
          \caption{Residual 1-norms for the two momentum components, turbulent viscosity and energy throughout the computation for the turbulent transonic airfoil case.
          Values are normalised by the initial residual.}
          \label{fig:rae_residuals}
        \end{figure}

        \paragraph{}
        The global flow is shown in figure \ref{fig:rae_field} through the pressure field near the airfoil.
        The shock is clearly seen on the upper part.
        The two computations give similar results: they are indistinguishable by just looking at the pressure field.
        This is to be expected as the traditional method gives already satisfying results on such applications.
        The pressure coefficient around the airfoil is given in figure \ref{fig:rae_cp}.
        Once again, the two curves are indistinguishable.
        Finally, the aerodynamic coefficients, the drag and lift coefficients, are given throughout the simulation in figure \ref{fig:rae_coefficients}.
        This time, the JFNK method gives an improvement: the coefficients are much more stable, which means the convergence is better.
        The blue curve corresponding to the traditional method struggles to converge and seems to oscillate periodically.
        In comparison, the oscillation of the orange curve is dampened: this corresponds to convergence.
        The scale of the oscillation seen in figure \ref{fig:rae_coefficients} is not significant physically speaking, as it is negligible.
        It shows an issue with the traditional implicit method of the solver CEDRE: it fails to reach proper convergence but it oscillates around the solution.
        The new JFNK method however achieves proper convergence.
        Even if the figure shows an advantage of the new method, the result is almost meaningless as the oscillation of the blue curve is numerically insignificant.
        Looking at other data is necessary to properly conclude on the convergence of the two methods.

        \paragraph{}
        The residual is the best way to measure if a method can help the convergence of the solver.
        To define the residuals, let us use the partial differential equation (\ref{eq:pde}).
        The residual is in fact the value of the function $\operatorname{F}$ from this equation.
        The name is quite adequate: for steady problems, the residual is what is left and still needs to be removed to find a steady solution.
        To decide on the convergence of a steady simulation, this residual norm is measured throughout the computational domain $\mathcal{D}$.
        More precisely, this residual is analysed component by component.
        For example, the residual 1-norm associated with the $i$th component is:
        \begin{equation}
          \norm[1]{\operatorname{F}_i} = \int_\mathcal{D} \left| \operatorname{F}_i \right| \mathrm{d}v
        \end{equation}
        and the $\infty$-norm is:
        \begin{equation}
          \norm[\infty]{\operatorname{F}_i} = \max_\mathcal{D} \left| \operatorname{F}_i \right| .
        \end{equation}
        In this fluid dynamics application, the residual norm is associated with the conservative variables: the density, the momentum components, the energy and the turbulent eddy viscosity $\nu_t$.
        With the Spalart--Allmaras compressible model, the conservative variable is in fact the product of the density and the Spalart--Allmaras variable $\rho \tilde{\nu}$.

        \paragraph{}
        Figure \ref{fig:rae_residuals} shows the residual norms from both computations.
        The gain from using the JFNK method appears clearly.
        The final residual norm is much smaller for each and every conservative component.
        This means that the Jacobian-Free Newton--Krylov method is able to reach a better convergence level than the traditional method.
        In other words, the better Jacobian matrix approximation leads to a better convergence than the older poor approximation.

        \paragraph{}
        In this first test case, we looked at the residual norms as the fields computed by both methods were almost identical.
        Moreover, our method was even able to reach better convergence levels.
        This validates our method on a first simple case, albeit with turbulence modelling and a shock.


    \pagebreak
    \subsection{Turbulent transonic airfoil with boundary layer resolution}

      \subsubsection{Definition of the test case}

        \paragraph{}
        The previous case uses a mesh that is too coarse to resolve the boundary layer but instead uses a turbulent wall model.
        It is representative of CEDRE applications, but not of actual applications from the aerodynamic community.
        The RAE 2822 airfoil is often used for turbulence model validations, but computations in the field prefer to use a finer mesh to resolve the boundary layer.
        This is what we present next.
        The new mesh, shown in figure \ref{fig:rae_mesh_fine} is clearly different.
        It is a three-dimensional mesh, which is in fact a two-dimensional mesh that was extruded on 6 cells in the $y$ direction.
        Periodicity is set between the two parallel sides of the domain.
        The mesh is made of \num{198144} cells whereas the previous RAE 2822 mesh was only made of \num{36541} cells.
        With these additional cells, this mesh is able to be fine on the wall boundary condition as seen in figure \ref{fig:rae_mesh_fine}.
        This means that the use of the turbulent wall model is no longer necessary, unlike in the previous test case.
        Turbulence closure is still provided by the Spalart--Allmaras one equation turbulence model.
        The scale of the mesh is the same: the chord is still equal to $1\si{\meter}$.

        \begin{figure}
          \centering
          \includegraphics[width=\textwidth]{figures/rae_mesh_fine.png}
          \caption{Finer mesh for the RAE 2822 test case. Close up on the leading and trailing edges.}
          \label{fig:rae_mesh_fine}
        \end{figure}

        \paragraph{}
        The flight conditions are also modified to match some existing experimental conditions: the AGARD-AR-138 experimental database.
        The Mach number is then 0.734 and the angle of attack is $2.54\si{\degree}$, with a Reynolds number of \num{6.5e6}.
        This corresponds to \emph{Case 9} from the experimental conditions.
        The parameters are not exactly the same as they are corrected to account for the effects of the wind tunnel \cite{HellstromDavidsonRizzi1994}.


      \subsubsection{Analysis of the results}

        \paragraph{}
        As before, the computation starts from an initial state constant in the domain.
        Then the computation is continued for a few more iterations with the traditional implicit Euler method.
        This is to reach a state close enough to the steady solution more quickly before analysing the convergence of our methods.
        After that, the computations are restarted, with the traditional method on one side and the matrix-free method on the other.
        The residuals are given in figures \ref{fig:rae_residuals_fine_l1} and \ref{fig:rae_residuals_fine_linf}.
        Overall, it takes much more iterations to decrease residual norms for this test case compared to the previous one.
        It is due to the larger number of cells, and the presence of small flattened cells in the boundary layer.

        \paragraph{}
        Figure \ref{fig:rae_residuals_fine_l1} shows that the reduction in residual 1-norms with the standard method is not satisfying.
        Users usually expect the solver to reduce the order of magnitude of the residuals by 4 or 5, which is more than what our solver does.
        In contrast, the matrix-free method reduces residual norms much more efficiently and converges to a more precise solution.
        The quality of the solution computed by the matrix-free method is then better.

        \begin{figure}
          \centering
          \includegraphics{figures/rae_residuals_fine_l1.png}
          \caption{Residual 1-norms for the horizontal and vertical momentum components, turbulent viscosity and energy throughout the computation for the turbulent transonic airfoil with boundary layer resolution case.
          Values are normalised by the initial residual.}
          \label{fig:rae_residuals_fine_l1}
        \end{figure}

        \paragraph{}
        The $\infty$-norm of the residual is given in figure \ref{fig:rae_residuals_fine_linf} for the same simulations.
        This time, the traditional method is not able at all to reduce residual norms.
        It means there are some cells in the computational domain for which the method is not able to reduce the local residual.
        The JFNK method in orange does not have this issue and converges to a better solution.
        The width of the curve is only due to data noise, as all lines use the same width.

        \begin{figure}
          \centering
          \includegraphics{figures/rae_residuals_fine_linf.png}
          \caption{Residual $\infty$-norms for the horizontal and vertical momentum components, turbulent viscosity and energy throughout the computation for the turbulent transonic airfoil with boundary layer resolution case.
          Values are normalised by the initial residual.}
          \label{fig:rae_residuals_fine_linf}
        \end{figure}

        \paragraph{}
        The value of the turbulent variable $\tilde{\nu}$ in the first cell above the wing profile is provided for both computations in figure \ref{fig:rae_field_fine}.
        The top views show the upper part of the airfoil, and the discontinuity due to the shock is visible, as expected.
        The bottom views show the lower part of the airfoil.
        The views on the right correspond to the JFNK computation, whereas the ones on the left to the computation with the traditional method.
        This traditional method using an approximated Jacobian matrix produces a solution of lower quality, as the pattern is not physical.
        In particular, there are cells for which $\tilde{\nu}$ is null in the computational domain, which should not happen with the standard Spalart--Allmaras model.
        Figure \ref{fig:rae_residuals_fine_linf} shows that the $\infty$-norm of the residual on the conservative variable $\rho \tilde{\nu}$ normalised by the initial residual is near $10^0$, so figure \ref{fig:rae_field_fine} highlights cells for which the residual is higher than $10^{-0.1}$.
        Those cells correspond to the ones for which $\tilde{\nu} = 0$.
        It shows a flaw in the standard time integration method, as it has non-physical features in the solution that prevent residual convergence.
        In contrast, it shows the quality of the JFNK method on this application.

        \begin{figure}
          \centering
          \includegraphics[width=\textwidth]{figures/rae_field_fine.png}
          \caption{
            Spalart--Allmaras variable $\tilde{\nu}$ on the RAE 2822 wing profile for the traditional method (left) and the JFNK method (right).
            The top views show the profile from above, and the bottom views from below.
            The highlighted region in green corresponds to cells where the absolute value of the residual on $\rho \tilde{\nu}$ is higher than $10^{-0.1}$.
          }
          \label{fig:rae_field_fine}
        \end{figure}


    \subsection{Hypersonic reactive sphere}

      \paragraph{}
      The second application we selected to compare the Jacobian-Free Newton--Krylov with the traditional one is the computation of the flow around a hypersonic solid sphere.
      Because of the high energy of the surrounding flow, the air molecules can separate and even form a plasma.
      The hypersonic reactive sphere is a well-known case, for both experimental \cite{Lobb1964} and numerical studies \cite{DobrovGimadievKarpenkoEtAl2022}.
      This is a simple yet representative test case of CEDRE applications.
      It then makes a lot of sense to analyse our new method on it.

      \subsubsection{Definition of the test case}

        \begin{figure}
          \centering
          \includegraphics[width=\textwidth]{figures/sphere_fields.png}
          \caption{Mesh along with final pressure, temperature, and \ce{NO} and \ce{e} mass fractions for the hypersonic reactive sphere test case.}
          \label{fig:sphere_fields}
        \end{figure}

        \paragraph{}
        The two-dimensional mesh is shown in figure \ref{fig:sphere_fields}.
        It is a regular mesh made of quadrangles, with refinement in the radial direction at the wall boundary and the expected shock location.
        The refinement at the wall boundary helps to compute more precisely the physical phenomena that happen at said boundary.
        The refinement at the shock is a requirement of the spatial discretisation methods.
        In order to get a clean and slim shock, the cells near the shock must have a high aspect ratio.
        The mesh takes this into account and is made according to CEDRE best practices.
        This expected shock location is obtained from a previous computation, made with a coarser mesh.
        This mesh is used in a two-dimensional axisymmetric computation to get the flow around a three-dimensional solid sphere.

        \paragraph{}
        The solid sphere is modelled by an isothermal wall boundary condition.
        This is a representative choice as well, as the heat flux going through the wall is one of the main interests of such computation.
        Moreover, as it depends on the derivatives of the flow variables, it is usually harder to get a correct value for the heat transfer.
        A better convergence will lead to better derivatives, which will lead to better physical results for such case users.
        No turbulence model is used, as the flow is mostly laminar.

        \paragraph{}
        The main feature of this test case is that it simulates a hypersonic flow.
        At the left, the input boundary condition feeds air at Mach 15.
        This will induce a strong shock, meaning a strong discontinuity in the flow.
        It is indeed seen in figure \ref{fig:sphere_fields} on the pressure and temperature.
        As this is a typical application of our solver, it is of importance to us that the new method behaves well with such flow features.
        When going through a strong shock, the temperature of the flow will increase a lot.
        The flow is made of air, or a mixture of 77\% \ce{N_2} and 23\% \ce{O_2}.
        At the high temperature they reach after the shock, the molecules can decompose into \ce{N}, \ce{O}, and \ce{NO}.
        They can even get ionised.
        The choosen model uses 11 possible species: \ce{N_2}, \ce{O_2}, \ce{N}, \ce{O} and \ce{NO}, the corresponding cations \ce{N_2^+}, \ce{O_2^+}, \ce{N^+}, \ce{O^+} and \ce{NO^+} and the electrons \ce{e^-}.

        \begin{figure}
          \centering
          \includegraphics[width=\textwidth]{figures/sphere_carbuncle.png}
          \caption{Streamlines near the stagnation point for two Riemann solvers.}
          \label{fig:sphere_carbuncle}
        \end{figure}

        \paragraph{}
        We argued previously that our solver is quite complex and that it makes it hard to do correct computations for non-experimented users.
        This hypersonic reactive sphere is a perfect example of this statement.
        Even if this case looks simplistic, it should not be taken lightly.
        When using unfit spatial discretisation methods, we ended up with a well-known problem of a such simulation called carbuncle \cite{MacCormack2013}.
        This phenomenon does not come from physics but is solely due to numerical methods.
        It is shown in figure \ref{fig:sphere_carbuncle} from some earlier computations that were using a different mesh.
        Simply put, it creates a recirculation bubble downstream of the shock, near the stagnation point.
        This bubble tends to push the shock farther from the sphere, and modify greatly the heat flux at this location.
        The choice of the Riemann solver ended up being the key element to getting rid of this undesirable effect: there is no recirculation with the AUSM+ scheme.
        There are other pitfalls regarding this test case in CEDRE.
        For example, the user can choose whether to interpolate the mass fraction $y_i$ for the species at the cell interfaces for the MUSCL scheme or the mass concentration $\rho y_i$.
        Choosing the default option leads to nonconverging residuals no matter the time integration method.
        Indeed, this default option is recommended when running simulations with multiple phases, with significant variations in the density.
        Because CEDRE is made to solve a large variety of problems, it has a lot of methods that come with a lot of parameters, so even a simple simulation such as this one requires a lot of knowledge if a good convergence is required.

        \paragraph{}
        This test case is a typical application of CEDRE, as opposed to the previous one which focuses more on the aerodynamic properties.
        Indeed CEDRE does not try to be a pure aerodynamic solver but a multiphysics one, equipped to solve high-energy problems.
        Reentry phenomena are therefore in the scope of our solver.
        Any improvements coming from our method on such applications would benefit a lot of CEDRE users on their applications.


      \subsubsection{Analysis of the results}

        \paragraph{}
        Once again, a first computation is done using the more robust traditional method.
        Indeed, the flow in the first cell against the sphere and the symmetry axis is initialised with a Mach number of 15 going straight into the wall.
        The matrix-free approximation struggles with such stiff initialisations, which is why the computation must start with another method, just until the shock has started to detach from the wall.
        Practically this amount to just a few iterations, compared to the number required to achieve convergence.

        \paragraph{}
        We first look at the different fields from the two computations.
        Once again they are similar and the residual norms are used to compare them, and this shows the simulation accounts for the features of interest.
        In figure \ref{fig:sphere_fields}, we see that the shock is present and that it falls as expected in the refined region.
        We see in the same figure the mass fraction of the various species downstream of the shock, which means the simulation is actively computing the chemical features of the flow, as expected.

        \paragraph{}
        To analyse the result from this case, one might once again look at the residuals.
        The first result corresponds to a computation without a reactive model.
        This simplification makes sense as the case to better understand the behaviour of the methods on an easier problem in terms of numerical complexity.
        It also means a different mesh is used, as not accounting for ionisation means the shock location is slightly different, but the two meshes are built the same way.
        The residual 1-norms are shown in figure \ref{fig:sphere_residuals}.
        It shows that the new method gives a better convergence.
        The difference between the traditional method and the new one is that the traditional method uses the Jacobian matrix of the first-order spatial discretisation method, whereas the new one approximates the true Jacobian matrix using the approximation (\ref{eq:matrix_free}).
        This can indeed be verified.
        Because this simulation does not use complex models, the poor approximation used in the traditional model should use the Jacobian matrix of the first-order scheme.
        If the approximation (\ref{eq:matrix_free}) uses the first-order evaluation of $f$, then it should approximate this same Jacobian matrix.
        The evolutions of the residual norms from this method match the ones from the traditional method in figure \ref{fig:sphere_residuals}.
        It first validates the development of the matrix-free approximation, and that the unbridle method should then use the actual second-order Jacobian matrix.
        It also confirms that the traditional method uses the Jacobian matrix of the first-order spatial discretisation method.
        Finally, it validates the choice of $\varepsilon$ and the matrix-free approximation, as it is able to give the same result as when we use the Jacobian matrix.

        \paragraph{}
        With the new method, using the true function $f$ with a second-order lead to much smaller residual norms.
        The only difference between this method and the previously existing one is that the latter uses the Jacobian matrix of the first-order spatial discretisation method, but the JFNK method takes into account the MUSCL reconstruction when using the Jacobian matrix.
        Because it uses a better Jacobian matrix in the linear problem, it gives a better solution to the nonlinear solver, which gives in turn a better time integration.
        It confirms that using a better Jacobian matrix helps the overall convergence when solving steady problems and validates the choice of the Jacobian-Free Newton--Krylov method.

        \begin{figure}
          \centering
          \includegraphics{figures/sphere_residuals.png}
          \caption{Residual 1-norms for the horizontal and vertical momentum components, \ce{N_2} volumic mass and energy throughout the computation for the hypersonic non-reactive sphere case.
          Values are normalised by the initial residual.}
          \label{fig:sphere_residuals}
        \end{figure}

        \paragraph{}
        When adding the reactive model, the conclusion is not in favour of the JFNK method anymore.
        The evolution of the residual norms is shown in figure \ref{fig:sphere_reac_residuals}.
        This time, the JFNK method does not longer find smaller residuals than the traditional method.
        In fact, when comparing figures \ref{fig:sphere_residuals} and \ref{fig:sphere_reac_residuals}, it appears that the traditional method converges better, enough so that it converges better than the JFNK method.
        This appears to happen because the mesh quality is better in this computation, compared to the mesh used without a reactive model.
        However, understanding the mesh quality is hard even on this simple test case for the reconstruction methods used in CEDRE.
        This is under current investigation by the responsible team.
        Still, the difference in the convergence of both methods is relatively small.
        A possible interpretation of this result is that when the mesh is not ideal for the spatial discretisation methods, using the JFNK method leads to a better convergence as it uses more precise Jacobian matrices for the linear system.
        When the mesh quality is better, the JFNK method is slightly less converged.
        Overall, it is still an improvement in the convergence on such test cases.

        \begin{figure}
          \centering
          \includegraphics{figures/sphere_reac_residuals.png}
          \caption{Residual 1-norms for the horizontal and vertical momentum components, \ce{N_2} volumic mass and energy throughout the computation for the hypersonic reactive sphere case.
          Values are normalised by the initial residual.}
          \label{fig:sphere_reac_residuals}
        \end{figure}

    \paragraph{}
    In a previous part, we suggested that the poor quality of the Jacobian matrix used in our solver was an obstacle to achieving good convergence.
    We then proposed an improvement: the Jacobian-Free Newton--Krylov method using the matrix-free approximation (\ref{eq:matrix_free}).
    We compared this new method with the traditional one on simple yet representative applications of our solver.
    This comparison showed that indeed, using a better Jacobian matrix leads to a lower residual norm.
    The new method can be useful when looking at quantities that require precise convergence, for instance, the derivatives of the flow variables.
    However, the main drawback of this new method is its computational cost.
    This cost is not inherent to the method but comes from our solver properties as was discussed previously.
    The recommended usage is then to start a computation using the inexpensive traditional method, and then achieve better convergence using the more precise yet more expensive new method.


  \section{Using the matrix-free formulation with a new fluid model}

    \paragraph{}
    In the previous section, the new Jacobian-Free Newton--Krylov method was compared to the traditional method that uses the first-order Jacobian matrix.
    This first-order matrix is easily computed for the usual Navier--Stokes equations.
    But as CEDRE is under constant development, new models are added in order to handle more finely various multiphysics phenomena.
    For example, a new model was recently added to better handle multiphasic flows \cite{Cordesse2020}.
    \PS{Another model that we will investigate in the following was added to better account for thermodynamic disequilibriums.}
    \footnote{
    \LM{mais du coup tu ne l'utilises pas du tout ce nouveau modèle dans la suite, non?}
    \PS{Nope c'est juste pour montrer que c'est quelque chose qui se fait, et donc que le Matrix free peut intéresser d'autres personnes.}
    }
    Adding a new model amount to writing the function $\operatorname{F}$ from equation (\ref{eq:pde}), or equivalently $\operatorname{G}$ from equation (\ref{eq:ode}).
    Usually, the development of this function is straightforward as it is given explicitly from the equations.
    However, the development of the Jacobian matrix is much more difficult.
    This is why the new models can not yet compute Jacobian matrices, and therefore can not use implicit methods as of today.
    The Jacobian-Free Newton--Krylov method is then a nice way to use implicit methods as it does not require the Jacobian matrix.
    Because the matrix-free approximation is written in a generic fashion, it can be used on any fluid model.
    In the following, we will use the JFNK method on a new fluid model that does not have an available Jacobian matrix and therefore can not use traditional implicit methods.


    \subsection{Multi-TEmperature model}

      \paragraph{}
      The new model we will use is called the \emph{Multi-TEmperature model}, or MTE.
      Its difference from the traditional Navier--Stokes model is that it allows for a thermodynamic non-equilibrium of some flow components.
      A particle has multiple degrees of freedom: translation, rotation and vibration.
      In more traditional models, it is assumed that all modes are at equilibrium, and they are grouped in what we call energy.
      One can define a time constant for each energy mode that corresponds to the number of collisions that are required to get to the equilibrium.
      A few collisions are required for translation modes, and about ten for rotation modes.
      This gives a small time constant of order $10^{-9} \si{\second}$.
      For vibration modes, however, up to \num{20000} collisions are required.
      The corresponding time constant is significantly larger, and there may be regions where there can be a disequilibrium between vibrational energy on one side and translation and rotation energy on the other.
      The energy of such components can no longer be described with a single temperature.
      As electrons are much lighter, they move more than the other heavier flow components.
      Their transitional energy and the transitional energy of other components can not be described with the same temperature.
      With the new Multi-TEmperature model, the flow components may be divided into three classes:
      \begin{itemize}
        \item the ones that always are at the equilibrium
        \item the ones that may be at vibrational disequilibrium
        \item the electrons that are handled separately from the other heavy component.
      \end{itemize}

      \paragraph{}
      To account for the disequilibrium, the Navier--Stokes equations (\ref{eq:ns}) are modified into the following:
      \begin{equation}\label{eq:mte}
        \left\{\begin{alignedat}{5}
          &\partial_t\left(  \rho_s  \right) &&+ \nabla\cdot\left(  \rho_s \vec{u}  \right) &&=
            \nabla\cdot\bigl( \rho D_s \nabla y_s\bigr) + \dot \omega_s \\[10pt]
          %
          &\partial_t\left(  \rho \vec{u}  \right) &&+ \nabla\cdot\left(  \rho \vec{u} \otimes \vec{u}  +  \left(p + p_e\right) \mat{\operatorname{Id}}  \right) &&=
            \nabla\cdot\biggl(
              \mu \left(\nabla \vec{u} + \nabla \vec{u}^T\right)
              - \frac{2}{3} \mu \left(\nabla \cdot \vec{u} \right) \mat{\operatorname{Id}}
            \biggr) \\[10pt]
          %
          &\partial_t\left(  \rho_m e_{v,m}  \right) &&+ \nabla\cdot\left(  \rho_m e_{v,m} \vec{u}  \right) &&=
            \nabla\cdot\bigl(
              \lambda_{v,m} \nabla T_{v,m}
              + \rho e_{v,m} D_m \nabla y_m
            \bigr) \\
            &&&&& \phantom{= {}} + S^{v-t}_m + S^{v-v}_m + S^{v-e}_m + \dot \omega_m e_{v,m} \\[10pt]
          %
          &\partial_t\left(  \rho_e e_e  \right) &&+ \nabla\cdot\left(  \left(\rho_e e_e + p_e\right) \vec{u}  \right) &&=
            \nabla\cdot\bigl(
              \lambda_e \nabla T_e
              + \rho h_e D_e \nabla y_e
            \bigr)
            + \vec{u} \cdot \nabla p_e \\
            &&&&& \phantom{= {}} + S^{e-t} + S^{e-r} + S^{e-v} + \dot \omega_e e_e \\[10pt]
          %
          &\partial_t\left(  \rho E  \right) &&+ \nabla\cdot\left(  \left(\rho E + p + p_e\right) \vec{u}  \right) &&=
            \nabla\cdot\Biggl(
              \lambda_{eq} \nabla T
              + \sum_m \lambda_m \nabla T_m
              + \lambda_e \nabla T_e \\
              &&&&& \phantom{=\nabla\cdot\Biggl(} + \mu \left(\nabla \vec{u} + \nabla \vec{u}^T\right) \vec{u} - \frac{2}{3} \mu \left(\nabla \cdot \vec{u}\right) \vec{u} \\
              &&&&& \phantom{=\nabla\cdot\Biggl(} + \rho \sum_s h_s D_s \nabla y_s
            \Biggr)
        \end{alignedat}\right.
      \end{equation}
      The first noticeable feature is that the equations of this physical model are more complex compared to the standard Navier--Stokes equations.
      Furthermore, there are multiple energies corresponding to the multiple temperatures: $e_e$ the energy of the electronic gas, $e_{v, m}$ the vibrational energy for each flow component that may be at disequilibrium, and $E$ the total energy.
      It increases the number of conservative variables.
      The variables $y$ correspond to the mass fraction of the different flow components.
      The source terms $\omega$ correspond to chemical production or decay.
      The source terms $S$ correspond to the energy transfers between the different energy modes.
      The details of those terms can be complex and are not discussed here as it is not the subject of this thesis, but can be found in \cite{Soubrie2006}.
      A noticeable feature of this model is the presence of a nonconservative term in the conservation equation of the electrons' energy: $\vec{u} \cdot \nabla p_e$.
      This term is due to the effect of the electric field.
      This is a known issue of this model, as the Finite Volume method handles better conservative terms.
      There are two ways of handling this issue.
      The first one uses electronic entropy instead of energy as a conservative variable \cite{CoquelMarmignon1995}.
      To do so, it makes some simplifications concerning electronic dissipation terms.
      However, this simplification can lead to inaccuracy in the physical result, and it appears that using the nonconservative formulation is necessary to get physically accurate results \cite{Soubrie2006, KimGuelhanBoyd2012}.
      The other approach is the one used in our model, which handles the nonconservative term from equation (\ref{eq:mte}).
      Not to go into too much detail, such models that account for thermodynamic disequilibrium are still under discussion as of today \cite{BlancoJosyula2020}.

      \paragraph{}
      The Multi-TEmperature model allows for non-equilibrium between the vibrational mode of the flow components, the electronic energy and the total energy.
      Taking this non-equilibrium into account helps to compute more precisely physical quantities such as the electronic density, which can be useful for magnetohydrodynamics applications for example.
      It gives a better prediction of shock layer radiation with a better representation of the population of energy states which helps the simulation of radiation cooling or the computation of wall heat fluxes.
      This model is most interesting on reentry problems, as there is indeed some non-equilibrium downstream of the strong shocks that appear on such problems.
      Taking the non-equilibrium into account can change the flow downstream of the shock and on the solid wall, and give more precise temperatures and chemical compositions.
      Generally speaking, the Multi-TEmperature model gives more precise results than the standard Navier--Stokes model on fast flow with low relative density where there may be some thermodynamic non-equilibrium, such as reentry problems or problems with the strong expansion of a plume.


    \subsection{Hypersonic reactive sphere}

      \paragraph{}
      The hypersonic sphere test case is a perfect fit for the Multi-TEmperature model.
      There is some thermodynamic non-equilibrium in the region downstream of the shock, because of its intensity.
      Unfortunately, the Jacobian matrix for the new MTE model is not currently available.
      This limits the MTE model users to explicit time integration methods, and therefore small time steps for stability reasons.
      As they deemed explicit time integration too slow, they wanted to use the newly implemented Jacobian-Free Newton--Krylov method for their simulations.
      We will show in the following another interest of the matrix-free approximation: it allows for implicit time integration methods despite having no Jacobian matrix.
      The goal here is to be able to get a steady solution to the problem as fast as possible for the user.
      Initially, the MTE model users were using a second-order Runge--Kutta method: the Midpoint method.
      We will compare the JFNK method to this reference method, and we will look in particular at how much time is required from the user to get to a physically satisfying result.


      \subsubsection{Full reactive model}

        \paragraph{}
        For the first step of our analysis, we will use the most complete physical model.
        We use the same 11 flow components as before: \ce{N_2}, \ce{O_2}, \ce{N}, \ce{O} and \ce{NO}, the corresponding cations \ce{N_2^+}, \ce{O_2^+}, \ce{N^+}, \ce{O^+} and \ce{NO^+} and the electrons \ce{e^-}.
        We of course use the Multi-TEmperature model, and we consider that \ce{N_2} and \ce{O_2} may be at vibrational disequilibrium.
        Other components may in fact also be at disequilibrium and we could take them into account but the computational cost increase is not worth it.
        As they are less present, and they quickly get to the equilibrium relatively to \ce{N_2} and \ce{O_2}, accounting for their disequilibrium would not be significant in the results \cite{Park2006}.
        This amount to having four temperatures to describe the flow: $T_e$ for the electrons, $T_{v, \ce{N_2}}$ and $T_{v, \ce{O_2}}$ for the vibration modes of \ce{N_2} and \ce{O_2}, and $T$ for the total energy.

        \paragraph{}
        As before, the Jacobian-Free Newton--Krylov method needs some help at the beginning.
        This is why the computation starts with some iterations of the Midpoint method and uses the first-order Finite Volume method.
        Then, it continues with the JFNK method, still using the first-order Finite Volume method, to quickly get a good enough approximation of the expected solution.
        Finally, to get a finer result, the simulation switches the second-order spatial discretisation method.
        Local time-stepping based on the CFL number is used with the implicit simulation using the JFNK method.
        The GMRES method can not use preconditioners that require the matrix, so the only one available is the diagonal preconditioner based on the cell volumes.
        It is not ideal as it is extremely simple, and some physics-based preconditioner would be preferable \cite{KnollKeyes2004} but none were available to us at the time.
        The spatial discretisation method used is the Multislope method, as it appears more robust than the $k$-exact method with the Multi-TEmperature model.

        \paragraph{}
        The goal of this thesis is not the physical analysis of this test case so it will be left out, but according to the MTE model users, the results are in agreement with the literature.
        Figure \ref{fig:sphere_mte_temperatures} the various temperatures along the symmetry axis.
        This justifies the use of the Multi-Temperature model: the region downstream of the shock is indeed in a thermodynamic non-equilibrium state.

        \begin{figure}
          \centering
          \includegraphics{figures/sphere_mte_temperatures.png}
          \caption{Temperature on the symmetry axis for the Multi-TEmperature model.}
          \label{fig:sphere_mte_temperatures}
        \end{figure}

        \paragraph{}
        We will also look at the residuals for this analysis.
        But this time, using the number of iterations as the $x$-axis would be unfair to the explicit method.
        Indeed the cost of one iteration is a lot smaller for the explicit method than for the implicit JFNK one.
        What matters to us and a typical user is not the number of iterations but the time spent waiting for the result.
        It is called the elapsed real time or wall-clock time.
        As the computations run each time in the same parallel environment, with a fixed number of CPU cores, the elapsed time is proportional to the CPU time.
        The $x$-axis will then be the elapsed real time: the actual time it took to get to the current residual.

        \begin{figure}
          \centering
          \includegraphics{figures/sphere_mte_residuals.png}
          \caption{Residual 1-norms for the electron mass fraction, vertical momentum, \ce{N_2} vibrational energy and total energy throughout the computation for the hypersonic reactive sphere case using the Multi-TEmperature model.
          Values are normalised by the initial residual.}
          \label{fig:sphere_mte_residuals}
        \end{figure}

        \paragraph{}
        Figure \ref{fig:sphere_mte_residuals} shows the residual norms for the two computations.
        The leftmost black curve corresponds to the initialisation: a few iterations of the explicit time integration method.
        From then the computation can continue with the Midpoint method on one side and the Jacobian-Free Newton--Krylov method on the other.
        We see that the residual norm obtained with the JFNK method is lower than its explicit counterpart.
        Indeed, the explicit method is limited to small time steps because of the shock and the stiff reactive model.
        The implicit method however is not.
        The fact that the implicit method can use larger time steps is not new, but with this specific model, it was impossible to use implicit methods until the matrix-free approximation was added.
        The JFNK method even ends up being cheaper in terms of CPU time.
        This means that for a user, it is faster to use the newly implemented Jacobian-Free Newton--Krylov method than to use explicit methods.
        Our method gives a better alternative to users working on problems with the Multi-TEmperature model.


      \subsubsection{Simplified reactive model}

        \paragraph{}
        The equations concerning electrons in the Multi-TEmperature model bring a lot of issues.
        It is because the quantity of free electrons is usually relatively small when compared to other flow components.
        It can lead to numerical instabilities or cause other problems.
        As the equations hold everywhere in the numerical domain, they hold in particular in areas without free electrons, upstream of the shock in our hypersonic sphere for example.
        The question one must ask is then how to compute quantities such as the electronic temperature $T_e$ where there are no free electrons.
        We will not go into such implementation details as it goes beyond the scope of this work, but the point is that having electrons is quite troublesome for numerical methods.
        According to users that work on such reentry applications, it is not always useful to compute fluid ionisation.
        We explained why it might be crucial for some simulations, but the impact of ionisation on other quantities is almost negligible when the interest is not on the electrons.
        Sometimes, disregarding degrees of freedom corresponding to the electrons leads to simpler yet meaningful computations.
        The corresponding computation is drastically simplified as the contribution of the electrons is difficult to handle.
        The result however still gives interesting information, when one is not interested in the effects of ionisation.
        It is done by considering only the chemical components \ce{N_2}, \ce{O_2}, \ce{N}, \ce{O} and \ce{NO}.

        \paragraph{}
        We decided to do the same comparison as before but this time with the simplified reactive model.
        The computation starts once again with a few iterations of the Midpoint method.
        Then, we compare the explicit method with the Jacobian-Free Newton--Krylov method.
        We show the corresponding residual norms in figure \ref{fig:sphere_tv_residuals}, while still using the elapsed real time as the $x$-axis.
        The result is similar to the previous one, except it is even more in favour of the JFNK method.
        The implicit method reaches much lower residual norms than the explicit one.
        With the full reactive model, the limiting factor for the implicit method came from the stiffness due to the electrons.
        \PS{Here, the term stiffness corresponds to both stiffness from the equation and difficulties arising from the implementation, that uses tools such as \emph{min/max} functions that add non-differentiability.}
        \footnote{
        \LM{On est sûr que c'est la raideur, ou c'est le fait que l'on patche, lorsqu'il n'y a plus d'électrons, par un raccord non C1?}
        \GP{Lionel, je comprends pas ce que tu sous entend... Il faudra qu'on clarifie après les vacances...}
        }
        Without them, the convergence is significantly faster.
        However, the time step for the explicit method is under the same limitations with both reactive models.
        The gain in CPU time is then huge, in favour of the JFNK method.

        \begin{figure}
          \centering
          \includegraphics{figures/sphere_tv_residuals.png}
          \caption{Residual 1-norms for the electron mass fraction, vertical momentum, \ce{N_2} vibrational energy and total energy throughout the computation for the hypersonic reactive sphere case using the Multi-TEmperature model without ionisation.
          Values are normalised by the initial residual.}
          \label{fig:sphere_tv_residuals}
        \end{figure}


  \paragraph{}
  This chapter compared the performances of the Jacobian-Free Newton--Krylov method with already existing methods of CEDRE on typical applications.
  When compared to the traditional implicit method on turbulent RANS computations, it showed that the JFNK achieves better convergence in the residual norms, when the convergence of the traditional method is unsatisfactory.
  The conclusion is similar when working with typical reentry applications.
  When the traditional method can converge well, however, the result is no longer in favour of the JFNK method, although the difference between the two methods is not significant.
  Finally, the matrix-free method was used with a newly implemented fluid model that does not give access to a Jacobian matrix yet.
  For this reason, using the JFNK method is the only way of using an implicit method.
  It was compared to the explicit Midpoint method that is currently used for computations with this fluid model.
  The results are once again in favour of the JFNK method, both in terms of convergence and speed.
  The superiority of the matrix-free method depends on the model's complexity level: the required time to reach the same convergence in residual norms is divided by 2 with the full model, and by 4 without ionisation.
